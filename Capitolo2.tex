L'odierna rete elettrica è stata progettata come un sistema centralizzato, in cui l'energia elettrica fluisce attraverso linee unidirezionali di trasmissione e distribuzione dai generatori fino ai clienti finali. La logica applicativa è concentrata in zona centrale e solo parzialmente nelle \emph{substations}, mentre le componenti restanti sono totalmente passive. Le Smart Grid forniscono una più elevata ed ampia intelligenza distribuita incorporata nei dispositivi locali, comunicazione e scambio bidirezionale di informazioni ed elettricità.
Tale nuovo approccio richiede sia una complessa infrastruttura di comunicazione, che sofisticate tecnologie di comunicazione e computazione. Entrambe consentono l'introduzione di nuovi metodi di gestione della domanda energetica e la conservazione di parte dell'energia prodotta, per adottare politiche di bilanciamento del carico.      


% http://ieeexplore.ieee.org/stamp/stamp.jsp?tp=&arnumber=6298960 
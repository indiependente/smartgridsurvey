%%%%%%%%%%%%%%%%%%%%%%%%%%%%%%%%%%%%%%%%%
% Large Colored Title Article
% LaTeX Template
% Version 1.1 (25/11/12)
%
% This template has been downloaded from:
% http://www.LaTeXTemplates.com
%
% Original author:
% Frits Wenneker (http://www.howtotex.com)
%
% License:
% CC BY-NC-SA 3.0 (http://creativecommons.org/licenses/by-nc-sa/3.0/)
%
%%%%%%%%%%%%%%%%%%%%%%%%%%%%%%%%%%%%%%%%%

%----------------------------------------------------------------------------------------
%	PACKAGES AND OTHER DOCUMENT CONFIGURATIONS
%----------------------------------------------------------------------------------------
%%%%%%%%%%%%%%%%%%%%%%%%%%%%%%%%%%%%%%%%%
% Code Snippet
% LaTeX Template
% Version 1.0 (14/2/13)
%
% This template has been downloaded from:
% http://www.LaTeXTemplates.com
%
% Original author:
% Velimir Gayevskiy (vel@latextemplates.com)
%
% License:
% CC BY-NC-SA 3.0 (http://creativecommons.org/licenses/by-nc-sa/3.0/)
%
%%%%%%%%%%%%%%%%%%%%%%%%%%%%%%%%%%%%%%%%%

\documentclass[11pt,oneside]{book}
%\documentclass[DIV=calc, paper=a4, fontsize=11pt, twocolumn]{scrartcl}	 % A4 paper and 11pt font size
%----------------------------------------------------------------------------------------

\usepackage{listings} % Required for inserting code snippets
\usepackage[usenames,dvipsnames]{color} % Required for specifying custom colors and referring to colors by name

%%%%%%%%%% Uncomment if you need to write some code
 \definecolor{DarkGreen}{rgb}{0.0,0.4,0.0} % Comment color
 \definecolor{highlight}{RGB}{255,251,204} % Code highlight color

 \lstdefinestyle{Style1}{ % Define a style for your code snippet, multiple definitions can be made if, for example, you wish to insert multiple code snippets using different programming languages into one document
 language=Java, % Detects keywords, comments, strings, functions, etc for the language specified
 backgroundcolor=\color{highlight}, % Set the background color for the snippet - useful for highlighting
 basicstyle=\footnotesize\ttfamily, % The default font size and style of the code
 breakatwhitespace=false, % If true, only allows line breaks at white space
 breaklines=true, % Automatic line breaking (prevents code from protruding outside the box)
 captionpos=b, % Sets the caption position: b for bottom; t for top
 commentstyle=\usefont{T1}{pcr}{m}{sl}\color{DarkGreen}, % Style of comments within the code - dark green courier font
 deletekeywords={}, % If you want to delete any keywords from the current language separate them by commas
 %escapeinside={\%}, % This allows you to escape to LaTeX using the character in the bracket
 firstnumber=1, % Line numbers begin at line 1
 frame=single, % Frame around the code box, value can be: none, leftline, topline, bottomline, lines, single, shadowbox
 frameround=tttt, % Rounds the corners of the frame for the top left, top right, bottom left and bottom right positions
 keywordstyle=\color{Blue}\bf, % Functions are bold and blue
 morekeywords={}, % Add any functions no included by default here separated by commas
 numbers=left, % Location of line numbers, can take the values of: none, left, right
 numbersep=10pt, % Distance of line numbers from the code box
 numberstyle=\tiny\color{Gray}, % Style used for line numbers
 rulecolor=\color{black}, % Frame border color
 showstringspaces=false, % Don't put marks in string spaces
 showtabs=false, % Display tabs in the code as lines
 stepnumber=5, % The step distance between line numbers, i.e. how often will lines be numbered
 stringstyle=\color{Purple}, % Strings are purple
 tabsize=4, % Number of spaces per tab in the code
 }

 % Create a command to cleanly insert a snippet with the style above anywhere in the document
 \newcommand{\insertcode}[2]{\begin{itemize}\item[]\lstinputlisting[caption=#2,label=#1,style=Style1]{#1}\end{itemize}} % The first argument is the script location/filename and the second is a caption for the listing
%%%%%%%%%%%%%%%%%%%%%%%%%%%%%

\usepackage[utf8x]{inputenc}
\usepackage{lipsum} % Used for inserting dummy 'Lorem ipsum' text into the template
\usepackage[english,italian]{babel} % English language/hyphenation
\usepackage[protrusion=true,expansion=true]{microtype} % Better typography
\usepackage{amsmath,amsfonts,amsthm} % Math packages
\usepackage[svgnames]{xcolor} % Enabling colors by their 'svgnames'
\usepackage[hang, small,labelfont=bf,up,textfont=it,up]{caption} % Custom captions under/above floats in tables or figures
\usepackage{booktabs} % Horizontal rules in tables
\usepackage{fix-cm}	 % Custom font sizes - used for the initial letter in the document
\usepackage{graphicx}
\usepackage{sectsty} % Enables custom section titles
\allsectionsfont{\usefont{OT1}{phv}{b}{n}} % Change the font of all section commands

%----------------------------------------------------------------------------------------
%	Clickable Links
\usepackage{hyperref}
\hypersetup{
    colorlinks,
    citecolor=black,
    filecolor=black,
    linkcolor=black,
    urlcolor=black
}
%----------------------------------------------------------------------------------------

\usepackage{fancyhdr} % Needed to define custom headers/footers
\pagestyle{fancy} % Enables the custom headers/footers
\usepackage{lastpage} % Used to determine the number of pages in the document (for "Page X of Total")
\usepackage{graphicx}
\usepackage{subfig}
% Headers - all currently empty
\lhead{}
\chead{}
\rhead{}

% Footers
\lfoot{}
\cfoot{}
\rfoot{\footnotesize Page \thepage\ of \pageref{LastPage}} % "Page 1 of 2"

\renewcommand{\headrulewidth}{0.0pt} % No header rule
\renewcommand{\footrulewidth}{0.4pt} % Thin footer rule

\usepackage{lettrine} % Package to accentuate the first letter of the text
\newcommand{\initial}[1]{ % Defines the command and style for the first letter
\lettrine[lines=3,lhang=0.3,nindent=0em]{
\color{DarkGoldenrod}
{\textsf{#1}}}{}}

\newcommand{\ggll}{\mathrel{\substack{\ll\\[-.05em]\gg}}} % convergence symbol

%----------------------------------------------------------------------------------------
%	TITLE SECTION
%----------------------------------------------------------------------------------------

\usepackage{titling} % Allows custom title configuration

\newcommand{\HorRule}{\color{DarkGoldenrod} \rule{\linewidth}{1pt}} % Defines the gold horizontal rule around the title

\pretitle{\vspace{-30pt} \begin{flushleft} \HorRule \fontsize{30}{30} \usefont{OT1}{phv}{b}{n} \color{DarkRed} \selectfont} % Horizontal rule before the title

\title{Smart Grid} % Your article title

\posttitle{\par\end{flushleft}\vskip 0.5em} % Whitespace under the title

\preauthor{\begin{flushleft}\large \lineskip 0.5em \usefont{OT1}{phv}{b}{sl} \color{DarkRed}} % Author font configuration

\author{Marco Amoruso, Daniele Anello, Francesco Farina, \\Iolanda Rinaldi } % Your name

\postauthor{\footnotesize \usefont{OT1}{phv}{m}{sl} \color{Black} % Configuration for the institution name
\newline
\newline
Università degli Studi di Salerno % Your institution

\par\end{flushleft}\HorRule} % Horizontal rule after the title

\date{Last edit: \today} % Add a date here if you would like one to appear underneath the title block

%----------------------------------------------------------------------------------------

\begin{document}

\maketitle % Print the title

\thispagestyle{fancy} % Enabling the custom headers/footers for the first page

%----------------------------------------------------------------------------------------
%	ABSTRACT
%----------------------------------------------------------------------------------------

% The first character should be within \initial{}

%----------------------------------------------------------------------------------------
%	ARTICLE CONTENTS
%----------------------------------------------------------------------------------------
\tableofcontents
\chapter{Introduzione}
L'infrastruttura elettrica attuale, non subisce modifiche da circa un centinaio d'anni: le componenti della rete, organizzate in una struttura gerarchica, sono vicine alla fine della loro vita. \newline Mentre la rete invecchia sempre più, la richiesta di energia elettrica aumenta gradualmente e l'attuale organizzazione è troppo complessa e poco adatta per soddisfare i bisogni del 21-esimo secolo. \newline Tra le mancanze dell'infrastruttura corrente vi sono: mancanza di analisi automatizzate, tempi di risposta lenti, mancanza di consapevolezza della situazione, ecc. A tali fattori, si uniscono anche l'aumento della popolazione sul pianeta e la conseguente richiesta di energia, il cambiamento del clima globale, i fallimenti delle apparecchiature che costituiscono la rete, i problemi di conservazione dell'energia, capacità limitata della generazione di energia, comunicazione unidirezionale, diminuzione di combustibili fossili e scarse capacità di recupero in caso di guasti. \newline È facile capire, analizzando tutti questi fattori, che vi è un bisogno urgente di una nuova infrastruttura elettrica in grado di risolvere tutti questi problemi. \newline \newline
L'attuale rivoluzione dei sistemi di comunicazione, particolarmente stimolata anche dalla crescita di Internet, offre la possibilità di migliorare il monitoraggio e, in generale, le funzionalità dei sistemi energetici e di rendere, quindi, le operazioni più efficaci e flessibili ma, allo stesso tempo, meno costose. \newline
La \textbf{Smart Grid} è l'opportunità per utilizzare le novità introdotte dall'ICT   (\textit{Information and Communication Technology}) al fine di rivoluzionare il sistema energetico. Tuttavia, a causa delle grandi dimensioni sia del sistema che dell'entità degli investimenti che sono stati fatti nel corso degli anni, qualsiasi cambiamento significativo sarà costoso e richiederà un'attenta giustificazione. \newline La Smart Grid è una rete elettrica moderna che offre migliore efficienza, affidabilità e sicurezza, permettendo anche una facile integrazione di nuove fonti di energia rinnovabile. \newline \newline
In confronto ai sistemi precedenti, la Smart Grid è concepita per integrare pienamente, all'interno di milioni di dispositivi, tecnologie che permettano comunicazioni veloci e bidirezionali e che permettano di formare un'infrastruttura dinamica ed interattiva con nuove e migliori capacità di gestione dell'energia. Tuttavia, una dipendenza così forte dal \textit{networking} di informazioni, inevitabilmente sottopone la Smart Grid ad una serie di potenziali vulnerabilità associate ai sistemi di comunicazione e di rete. Ciò, in pratica, aumenta il rischio di compromettere l'affidabilità e la sicurezza delle operazioni dell'infrastruttura che costituiscono gli obiettivi principali della Smart Grid. Per esempio, una potenziale intrusione nella rete da parte di un individuo non autorizzato, potrebbe portare ad una serie di conseguenze negative che vanno dall'\textit{information leakage} alla generazione di fallimenti in cascata, come ad esempio un blackout totale e la distruzione dell'intero sistema. \newline Pertanto, lo scopo di questa tesina è analizzare i problemi legati alla \textbf{sicurezza} della Smart Grid, che è critica nella progettazione delle reti di comunicazione ed è considerata una delle più alte priorità nello sviluppo della rete elettrica moderna.
\newline \newline
La tesina è strutturata nel seguente modo:
\begin{itemize}
\item Capitolo 2
\item Capitolo 3
\item Capitolo 4
\item Capitolo 5
\item Capitolo 6
\end{itemize}
\chapter{Smart Grid: motivazioni e definizione}
\section{Panorama energetico attuale}
La \textit{rete elettrica} attuale è il risultato di una rapida urbanizzazione e di un rapido sviluppo di infrastrutture in varie zone del mondo. Sebbene tali reti esistano ormai in molte aree geografiche diverse, le aziende generalmente tendono ad adottare tecnologie molto simili tra di loro. Ciononostante, restano altri fattori  di varia natura (economica, politica, geografica) legati allo sviluppo energetico che si diversificano a seconda dell'azienda. \newline
In generale però, pur tenendo in considerazione le differenze portate da tali fattori, la topologia base della rete elettrica attuale è rimasta immutata.
\begin{figure}[h] \centering{
\includegraphics[scale=0.3, natwidth=1003,natheight=490]{imgs/elect_grid.png}}
\caption{La rete elettrica attuale}
\end{figure}
\newline La struttura della rete attuale è una struttura strettamente gerarchica. La figura 1.1 mostra l'esistenza di tre sottosistemi distinti: generazione, trasmissione e distribuzione. \newline Le centrali elettriche sono composte da generatori elettromeccanici i quali, durante la fase di \textbf{generazione}, spinti dal flusso dell'acqua corrente o da motori termici alimentati da combustioni chimiche, generano energia. Tale energia viene successivamente inviata ai trasformatori del livello di \textbf{trasmissione}, i quali la convertiranno in energia ad alto voltaggio per permetterne la diffusione a lunga distanza. Dopo tale step, si passa alla \textbf{distribuzione}, in cui si applica prima una trasformazione a medio e basso voltaggio e, in seguito, si procede all'erogazione agli utenti finali.
\newline Tale sistema è basato sostanzialmente su una comunicazione \textit{unidirezionale} in cui la sorgente non ha nessuna informazione real-time circa le necessità degli ultimi punti della catena. Pertanto si tende a sovraccaricare la rete, facendole raggiungere a priori i picchi massimi di carico; poiché è raro che le richieste degli utenti raggiungano tali valori, questo approccio porta a rendere la rete elettrica un meccanismo inefficiente.
\newline
Inoltre, le reti elettriche attuali sono interconnesse tra loro a formare reti regionali o nazionali con lo scopo di fornire rotte ridondanti e alternative per il flusso della corrente in caso di problemi. \newline
La distribuzione dell'energia è gestita da un \textit{controllore centralizzato} che ha il compito di amministrare diverse regioni da un'unica posizione centrale. 
\newpage
\section{Cenni storici}
Le tecnologie connesse con le reti elettriche hanno radici che risalgono alla fine del XIX secolo: la \textit{corrente diretta} di Thomas Edison e la \textit{corrente alternata} di Nikola Tesla continuano ad essere utilizzate tutt'ora. Oggi, infatti, l'energia viene trasmessa utilizzando la corrente alternata, mentre quella diretta ha applicazioni specifiche, solitamente all'interno di plessi residenziali e commerciali. \newline \newline
Le principali topologie di rete elettrica attualmente in uso sono due: \textbf{radial grid} e \textbf{mesh grid}.


\begin{figure}[h]\centering{
  \includegraphics[scale=0.2, natwidth=754,natheight=909]{imgs/radialgrid.png}
  \caption{Radial grid}
}
\end{figure}

La radial grid (figura 1.2) è la topologia più comune; in essa l'elettricità viene distribuita a partire da una sottostazione seguendo un modello che ricorda un albero con molti rami e foglie.
Man mano che l'energia fluisce attraverso i cavi elettrici, la sua forza viene ridotta finchè non raggiunge la destinazione.

\begin{figure}[h]\centering{
  \includegraphics[scale=0.2, natwidth=529,natheight=867]{imgs/meshgrid.png}
  \caption{Mesh grid}
}
\end{figure}

La mesh grid (figura 1.3) fornisce una maggiore affidabilità rispetto alla radial grid poichè in quest'ultima ogni ramo ed ogni foglia ricevono energia da una singola sorgente (l'albero), mentre in una mesh l'energia può essere fornita attraverso varie fonti (altri rami e foglie). \newline
Le radial grid forniscono, inoltre, ridondanza limitata: in caso di malfunzionamento, una sottostazione vicina può entrare a far parte della rete, ma ciò presuppone che tale sottostazione non sia nelle stesse condizioni di quella che ha generato il guasto. \newline
Queste due topologie sono molto diffuse negli Stati Uniti.
Vi è, poi, una terza topologia utilizzata principalmente in Europa: \textbf{looped topology}, nata come insieme delle due reti precedenti (figura 1.4). 

\begin{figure}[h]\centering{
  \includegraphics[scale=0.2, natwidth=525,natheight=873]{imgs/looptop.png}
  \caption{Looped topology}
}
\end{figure}
\newpage
Tale topologia è molto simile alla radial grid, ad eccezione del fatto che, a partire dalla sottostazione, i rami e le foglie hanno due cammini separati. L'obiettivo della looped topology è essere capace di resistere alle rotture interne alla rete, indipendentemente dal punto in cui si verificano; per questo motivo tale topologia risulta essere più costosa rispetto alla radial grid.
\newline \newline
Con il passare del tempo si è sentita sempre di più la necessità di svecchiare tali topologie, tenendo in conto tanti fattori e tante motivazioni.\newline
Considerando, per esempio, che quasi il 90\% dei guasti alla rete elettrica provengono dal sottosistema di distribuzione, le modifiche e i miglioramenti partono proprio da quest'ultimo. In più, il rapido aumento dei costi relativi ai combustibili fossili insieme all'inabilità delle aziende di espandere la loro potenza di generazione mantenendola in linea con la domanda sempre più crescente, ha accelerato il bisogno di modernizzare la rete di distribuzione introducendo tecnologie capaci di gestire meglio sia le richieste che i guadagni ottenuti. \newline
\begin{figure}[h]\centering{
  \includegraphics[scale=0.2, natwidth=1117,natheight=714]{imgs/evolution.png}
  \caption{L'evoluzione della smart grid}
}
\end{figure}

La figura 1.5 mostra che gli investimenti degli ultimi anni si sono focalizzati principalmente sull'aspetto della rete elettrica che riguarda le misurazioni (\textit{metering}). \newline I primi progetti in questo settore hanno visto la nascita dei sistemi di \textbf{automated meter reading} (AMR) all'interno del sistema di distribuzione. \newline
L'infrastruttura AMR, nata nel 1977, ha introdotto l'automazione nella rete elettrica. Attraverso una combinazione di tecnologie, incluse reti wireless e wired,   AMR ha permesso alle compagnie di leggere le misurazioni da remoto, di ottenere le informazioni quasi in real-time e di fornire agli utenti bollette basate sui loro consumi reali (in precedenza le compagnie emettevano le bollette basandosi sulle stime dei consumi del cliente).  \newline Inoltre, attraverso questo meccanismo di recupero informazioni tempestivo, le aziende sono state capaci di migliorare la produzione di energia attraverso un maggiore controllo durante periodi di alta e bassa richiesta. \newline \newline
Sebbene la teconologia AMR all'inizio abbia attirato molta attenzione, le aziende presto si sono rese conto che non risolve il loro problema principale: la gestione demand-side. A causa della sua comunicazione unidirezionale, le capacità di AMR sono ridotte alla sola lettura dei dati e non è permesso, per esempio, modificare il comportamento della rete a seconda delle informazioni ricevute. \newline Pertanto AMR ha avuto vita breve; le aziende, piuttosto che continuare ad investire su di essa, hanno preferito spostarsi verso l'\textbf{advanced metering infrastructure} (AMI). \newline
L'AMI (di cui si parlerà in dettaglio nel capitolo 2), è un'architettura che permette la comunicazione automatizzata e bidirezionale tra uno smart meter e una società di servizi. L'obiettivo è quello di fornire a tali società informazioni real-time circa i consumi energetici e permettere agli utenti di fare scelte consapevoli sull'utilizzo dell'energia basate sui costi all'istante di utilizzo.
\newline
\newline
Il passo successivo nell'evoluzione della distribuzione della corrente elettrica è costituito dalla \textbf{Smart Grid}, che utilizza l'AMI come componente core per il recupero delle informazioni circa lo stato della rete e i consumi utente.
\newpage

\section{Perchè la Smart Grid}
Le industrie del settore dei servizi pubblici di tutto il mondo attualmente cercano di risolvere numerosi problemi, tra cui 
\begin{itemize}
\item diversificazione della generazione di energia
\item gestione delle richieste utente
\item conservazione dell'energia
\item riduzione globale dell'emissione di anidride carbonica
\end{itemize}
È evidente che tali problemi non possono essere risolti all'interno dei confini della rete elettrica esistente. \newline
Come detto in precedenza, la natura della comunicazione della rete attuale è unidirezionale. Essa converte solo un terzo di energia combustibile in elettricità, senza preoccuparsi del calore disperso. Circa l'8\% della corrente prodotta, viene dissipata poi attraverso i cavi elettrici, mentre il 20\% esiste solo per prepararsi ad un eventuale picco di carico di richieste utente (ed è in uso solo il 5\% delle volte). \newline
In aggiunta a tali problemi, vi è l'inadeguatezza della struttura gerarchica della rete. A causa di tale organizzazione, infatti, si ha quello che viene definito come \textit{effetto domino dei guasti}. \newline
La rete elettrica di nuova generazione, la Smart Grid, si propone di occuparsi delle maggiori carenze della rete attuale.
\begin{figure}[h]\centering{
  \includegraphics[scale=0.3, natwidth=669,natheight=553]{imgs/differences.png}
  \caption{Differenze tra la rete elettrica attuale e la Smart Grid}
}
\end{figure}

A partire dal 2005, c'è stato un interesse sempre più crescente verso le Smart Grid. Il riconoscere che l'ICT (\textbf{Information and Communication Technology}) offre significative opportunità per modernizzare il funzionamento delle reti elettriche unito alla consapevolezza che la produzione di energia può essere migliorata solo con un continuo ed efficiente monitoraggio, ha fatto si che si muovessero i primi passi verso la Smart Grid. In aggiunta a tali fattori, ci sono anche altre motivazioni a favore del passaggio verso una rete elettrica moderna.
\begin{itemize}
\item \textit{Strutture non più adeguate}: in molte zone del mondo (per esempio in USA e in alcuni paesi dell'Europa), i sistemi si sono rapidamente espansi a partire dal 1950 e le strutture relative alla trasmissione e alla distribuzione che furono installate a quel tempo non sono più adatte e hanno bisogno di essere sostituite. \newline Il bisogno di rinnovare tali componenti è un'ovvia opportunità di innovazione e, quindi, di introduzione di nuovi modelli e pratiche operative. \newline A ciò si aggiunge il fatto che, in molti paesi, i circuiti elettrici hanno bisogno di adattarsi ai carichi sempre più crescenti e all'introduzione di nuove fonti di energia rinnovabili. Ciò richiede, quindi, metodi più intelligenti sia per aumentare la capacità di trasmissione dell'energia, sia per reindirizzare il flusso di corrente verso circuiti meno carichi.

\item \textit{Vincoli termici}: tali vincoli si riferiscono ai limiti dei sistemi di trasmissione e distribuzione relativamente alla loro capacità di diffusione dell'energia. Quando le attrezzature trasportano la corrente eccedendo la loro potenza termica, si surriscaldano e i materiali atti all'isolamento si deteriorano rapidamente. Ciò comporta una riduzione della lifetime delle attrezzature e un aumento di possibilità fallimenti.\newline I vincoli termici dipendono molto dalle condizioni dell'ambiente esterno, che cambiano durante gli anni. Pertanto l'uso di rate di trasmissione e distribuzione dinamici può aumentare la capacità del circuito in qualsiasi momento.

\item \textit{Vincoli operativi}: qualsiasi sistema opera all'interno di predefiniti vincoli di voltaggio e frequenza. Se il voltaggio eccede il limite, i materiali isolanti dei componenti del sistema e le attrezzature degli utenti possono essere danneggiate, portando a corto-circuito. Per quanto riguarda la frequenza invece, gli operatori tendono a mantenerla all'interno di un range molto piccolo e, quando varia, intervengono dei servizi appositi con il compito di riportarla nell'intervallo prestabilito. \newline
La generazione di energia rinnovabile, però, ha un output variabile che non può essere previsto con certezza qualche ora prima. Pertanto mantenere il bilancio erogazione-richiesta e la frequenza del sistema nei limiti risulta essere un compito arduo. Per tale motivo gli operatori dei sistemi sono sempre alla ricerca di nuovi servizi per la gestione della frequenza. \newline Si pensa che in futuro l'utilizzo delle Smart Grid in vari ambiti, per esempio domestico e automobilistico, porterà ad avere carichi sempre più flessibili. Ciò aiuterà nel mantenere la stabilità della rete.

\item \textit{Sicurezza delle forniture}: la società moderna richiede una fornitura di energia che sia sempre più affidabile, man mano che carichi sempre più critici vengono connessi alla rete. L'approccio tradizionale per migliorare l'affidabilità, come visto in precedenza, era quello di installare cammini ridondanti, con notevole impatto sia sui costi che sull'ambiente. \newline L'approccio della Smart Grid in caso di guasti, invece, prevede l'utilizzo di intelligenti meccanismi di riconfigurazione, in modo tale da mantenere attiva la fornitura ai clienti ma evitando i costi addizionali portati da ulteriori circuiti.
\item \textit{Iniziative nazionali}: molti governi nazionali incoraggiano le iniziative delle Smart Grid poichè le vedono come un meccanismo redditizio ma allo stesso tempo economico per rinnovare le loro infrastrutture e, contemporaneamente, introdurre risorse rinnovabili.
\end{itemize}

\begin{figure}[h]\centering{
  \includegraphics[scale=0.3, natwidth=759,natheight=572]{imgs/benefits.png}
  \caption{Benefici introdotti dalla Smart Grid}
}
\end{figure}
\newpage
\section{Cos'è una Smart Grid}

\section{Tecnologie coinvolte}
\chapter{Architettura}
L'odierna rete elettrica è stata progettata come un sistema centralizzato, in cui l'energia elettrica fluisce attraverso linee unidirezionali di trasmissione e distribuzione dai generatori fino ai clienti finali. La logica applicativa è concentrata in zona centrale e solo parzialmente nelle \emph{substation}, mentre le componenti restanti sono quasi totalmente passive. Una Smart Grid, mostrata dal punto di vista strutturale in figura \ref{fig:1}, fornisce una più elevata ed ampia intelligenza distribuita incorporata nei dispositivi locali, comunicazione e scambio bidirezionale di informazioni ed elettricità.

\begin{figure}[h] \centering{
\includegraphics[scale=0.95]{imgs/architecture.jpg}}
\caption{Smart Grid}\label{fig:1}
\end{figure}

\section{Smart Grid Framework}
Le Smart Grid richiedono sia una complessa infrastruttura di comunicazione, che sofisticate tecnologie di comunicazione e computazione. Entrambe consentono la conservazione di parte dell'energia prodotta e l'introduzione di nuovi metodi di gestione della domanda energetica, per adottare politiche di bilanciamento del carico, controllare instabilità energetiche causate dalla natura delle risorse rinnovabili e prevenire la diffusione di fallimenti in cascata nella rete. 
\newline 
La figura \ref{fig:2} riassume le principali tematiche relative alle Smart Grid:
\begin{itemize}
	\item \emph{Energy infrastructure} rappresenta la base fisica ed organizzativa necessaria per la generazione, trasmissione e distribuzione dell'energia;
	\item \emph{Communication infrastructure} è responsabile del trasferimento di informazioni critiche attraverso la rete;
	\item \emph{Information technology} fornisce modelli, analisi, visualizzazioni web e transazioni commerciali;
	\item \emph{Potential applications} offre tecniche di generazione, gestione, automatizzazione e rilevamento per l'intero sistema.
\end{itemize} 

\begin{figure}[h] \centering{
\includegraphics[scale=0.3, natwidth=1003,natheight=490]{imgs/ict.png}}
\caption{Smart Grid framework}\label{fig:2}
\end{figure}

La communication infrastructure svolge un ruolo cruciale, ossia collegare tutte le componenti della rete collezionando informazioni sulle loro condizioni, per scopi di controllo, monitoraggio e manutenzione. Eventuali problemi legati all'energy instrastructure possono essere evitati se corrette operazioni vengono prese con l'aiuto della communication infrastructure. 
\newline 
Differenti tecnologie di comunicazione posso essere usate per diversi scopi e requisiti in base all'applicazione. L'information technology fornisce una piattaforma comune di scambio di informazioni proveniente da differenti attività legate alla Smart Grid, che permette l'integrazione di informazioni da diversi livelli, dando sostegno alla raccolta di diverse informazioni, all'analisi e ad applicazioni avanzate.
\newline
Le tecniche dell'applications layer generalmente mirano a ridurre il consumo energetico dei clienti, cambiando i loro comportamenti di consumo, dotandoli di strumenti di monitoraggio.    
\newline
La figura \ref{fig:3} mostra le componenti della Smart Grid, illustrate dall'energy infrastructure al potentional applications.

\begin{figure}[h] \centering{
\includegraphics[scale=0.7, natwidth=1003,natheight=490]{imgs/sgframework.png}}
\caption{Smart Grid framework}\label{fig:3}
\end{figure}

%\section{Communication architecture}
\newpage
Il concetto di Smart Grid mira a realizzare un sofisticato sistema, integrando information technology e communication infrastructure all'attuale sistema di alimentazione e il nuovo sistema di generazione distribuito, in modo da sfruttare pienamente l'uso di risorse rinnovabili e di massimizzare l'efficienza energetica. Da una prospettiva leggermente diversa, una Smart Grid può essere considerata come una rete di comunicazione di dati che riesce, grazie al supporto di specifici dispositivi di gestione dell'energia, a far collaborare le diverse componenti della rete in maniera flessibile e senza discontinuità, per un utilizzo efficiente dell'energia.
%\newline
%L'architettura \emph{end-to-end} delle Smart Grid (fig. \ref{fig:4}) fondamentalmente comprendono tre livelli principali:
%\begin{itemize}
%	\item \emph{Application layer}, include applicazioni avanzate fornendo interoperabilità fra di esse; si occupa principalmente della gestione della domanda/risposta, interruzioni, infrastruttura di metering, delle risose ed rilevamento di frodi;
%	\item \emph{Power layer}, comprende i sistemi di generazione, trasmissione e distribuzione, l'integrazione di risorse rinnovabili ed il sistema di comunicazione bidirezionale;  
%	\item \emph{Communication layer}, rappresenta il cuore del sistema fornendo la connettività fra tutte le parti e dispositivi di esso. 
%\end{itemize}     

%\begin{figure}[h] \centering{
%\includegraphics[scale=0.4, natwidth=1003,natheight=490]{imgs/endtoendtax.jpg}}
%\caption{Architettura end-to-end}\label{fig:4}
%\end{figure}

La comunicazione consiste di tre categorie di trasmissione, con relativi standard e protocolli (vedi cap.):
\begin{itemize}
	\item \emph{Wide-area network} (WAN);
	\item \emph{Field-area network} (FAN);
	\item \emph{Home-area network} (HAN).
\end{itemize}

%Il communication layer consiste di tre categorie di trasmissione:
%\begin{itemize}
%	\item \emph{Wide-area network} (WAN);
%	\item \emph{Field-area network} (FAN);
%	\item \emph{Home-area network} (HAN).
%\end{itemize}

\subsection{Wide-area network}
La WAN consente la comunicazione fra le entità che forniscono energia e le substations; deve estendersi su tutte le substation, strutture di distribuzione, generazione energetica e di conservazione dell'energia, per poter essere efficace e scalabile. Essa è rete di comunicazione bidirezionale ad alta larghezza di banda che gestisce le trasmissioni a lunga distanza dei dati con avanzate applicazioni di misurazione e monitoraggio. La comunicazione remota fra le \emph{utility} e gli smart meter è essenziale per lo scambio di importanti informazioni, quali prezzi e tariffe dei clienti. Le reti cellulari, WiMAX e comunicazioni cablate, in particolare comunicazioni basate su fibra ottica e microwave, sono i migliori candidati come tecnlogie per WAN (vedi cap. 4).
\newline
Il sistema di distribuzione agisce da punto di aggregazione fra FAN e WAN, come ad esempio una substation o una torre di comunicazione che colleziona tutte le informazioni prodotte dagli smart meter e le trasferisce alla rete di comunicazione principale. Oltre che da punto di aggregazione tali dispositivi possono fungere da punti di conservazione dell'energia per eventuali interruzioni o guasti.

\subsection{Field-area network}
La FAN può essere descritta come una rete di comunicazione per aree di distribuzione dell'energia e che mette in contatto l'automazione della distribuzione e dispositivi di controllo alle sedi dei consumatori. Essa agisce, quindi, come un intermediario fra le substation e le sedi dei clienti, con nodi intelligenti in grado di raccogliere e controllare i dati da remoto. Tali nodi sono connessi ad un gateway centralizzato, il quale è alimentato costantemente in modo da poter trasmettere i dati raccolti. I canali a bassa larghezza di banda della FAN sono altamente robusti per la trasmissione affidabile di dati. 
\newline 
La scelta delle tecnologie di comunicazione  variano per la FAN in base alle esigenze della Smart Grid: fibra ottica per avere bassa latenza e perfomance di comunicazione superiori, oppure WiMAX se le reti cellulari non riescono a coprire l'area di interesse, ma l'attuale orientamento ricade sull'utilizzo dello standard IEC 61850 (vedi cap. ), il quale fornisce interoperabilità e comunicazione fra i dispositivi elettronici intelligenti.

\subsection{Home-area network}{
Gli smart meter riescono a connettersi alla HAN, in modo tale che i consumatori siano in grado di conoscere l'importo da pagare e gestire il loro consumo ed avere il controllo dei propri elettrodomestici intelligenti, attraverso display presenti in casa e interfacce web. 
\newline 
Le migliori tecnologie di comunicazione per HAN sono ZigBee,Wi-Fi, HomePlug, Z-wave e M-Bus (vedi cap.).
}

\vspace{20pt}\hspace{-17pt}Nelle sezioni successive vengono presentate le tecnologie e le infrastrutture abilitanti di una Smart Grid, a partire dalla generazione e conservazione dell'energia fino ad arrivare alla trasmissione e distribuzione.

\section{Generazione di energia rinnovabile}
Le risorse di energia rinnovabile sono state sviluppate in molti paesi per ridurre l'inquinamento e fornire energia elettrica sostenibile. A differenza delle tradizionali fonti di energia, le quali creano inquinamento, le risorse di energia rinnovabile non esauriscono risorse naturali nel processo di creazione di energia e sono adattabili ovunque, in base alle dimensioni a partire dall'applicazione su una singola casa fino a dimensioni su larga scala.
\newline
Le più comuni risorse di energia rinnovabile sono:
\begin{itemize}
	\item \emph{Sistemi fotovoltaici}, i quali convertono l'energia solare direttamente in elettricità, attraverso pannelli solari esposti al sole. Tali pannelli sono costituiti da celle solari che contengono materiale fotovoltaico, le quali trasmettono elettroni tra diverse bande all'interno del materiale generando differenza di potenziale fra due elettrodi, che consente alla corrente continua di fluire;
	\item \emph{Sistemi per l'energia solare termica}, che convertono energia solare in calore. Esistono tre tipi di raccoglitori in base alla temperatura, da bassa per riscaldare piccoli spazi ad alta per l'utilizzo nella produzione di energia elettrica;
	\item \emph{Vento}, la cui energia viene convertita tramite turbine in elettricità. Il principale aspetto negativo deriva dall'intermittenza del vento, specularmente per i sistemi basati sull'energia solare;
	\item \emph{Biomasse}, ovvero la produzione di elettricità a partire da elementi naturali morti, anche se questa causa inquinamento atmosferico;
	\item \emph{Sistemi che sfruttano la potenza dell'acqua}, sia che essa sia generata artificialmente che naturalmente, grazie alle onde e alle maree.   
\end{itemize} 

 
\section{Conservazione dell'energia}
Il principale problema con l'energia elettrica è che deve essere utilizzata non appena viene generata, o in caso contrario, deve essere convertita in altre forme di energia. Durante i periodi in cui non è richiesta la loro assistenza, sistemi di stoccaggio accumulano energia. Successivamente, l'energia immagazzinata viene inviata nel sistema di alimentazione in determinati periodi di tempo, pertanto
riducendo la richiesta di generazione e assistendo il sistema quando necessario. 
Tali sistemi di conservazione dell'energia vengono sfruttati per diversi scopi:
\begin{itemize}
	\item Mitigare fluttuazioni e perdite momentanee di potenza;
	\item Gestire cambiamenti frequenti di richiesta energetica per garantire stabilità del sistema; 
	\item Sostenere l'intermittenza e a mancanza di controllabilità nella generazione di energia rinnovabile, fornendo l'energia mancante e sottraendo quella in eccesso rispetto alla domanda;
	\item Conservare energia in determinati periodi, ad esempio quando la domanda oppure il prezzo sono bassi e scaricarla quando conviene.
\end{itemize}

Storicamente le centrali idroelettriche sono state le più comuni applicazioni di stoccaggio dell'energia, tuttavia negli ultimi decenni sono state introdotte nuove tecnologie in tale ambito:
\begin{itemize}
	\item Batterie, in grado di immagazzinare energia durante le fasi di carico/scarico;
	\item Pile a combustibile, che permettono di ottenere energia mediante reazioni chimiche senza che avvenga alcun processo di combustione termica, a partire da ossigeno ed idrogeno;
	\item Volani, i quali possono accumulare energia cinetica in masse rotanti e rilasciarla rallentandone la rotazione;
	\item Superconduttori magnetici, capaci  di raccogliere energia in campi magnetici, che vengono creati attraverso il passaggio di corrente continua in super bobine.  
\end{itemize}

\section{Veicoli elettrici}

\section{Microgrid}



\section{Smart substation}

\subsection{IED}
\subsection{Sensor}
\subsection{SCADA}



\section{Sistemi di trasmissione}

\subsection{Sistemi di gestione dell'energia}
\subsection{FACTS}
\subsection{HVDC}
\subsection{WAMPAC}


\section{Sistemi di distribuzione}

\subsection{Sistemi di gestione della distribuzione}
\subsubsection{distribution SCADA}
\subsection{Volt/VAr control}


%
%\subsection{smart substation} 3.3 , 3.3.8 role
%IED, sensor, scada, RTU, 
%
%\subsection{energy storage}
%veicoli, microgrid (3.2.4)
%
%\subsection{generazione di energia da risorse rinnovabili}
%
%\subsection{trasmission systems} 3.4 (pag.154) role 209
%facts hdvc 3.4.2 (pag.168), WAMPC 3.4.3 (pag.187) 202 role
%
%\subsection{distribution systems} 3.5 pag.211
%distribution scada, Fault Detection, Isolation, and Service Restoration 244, componenti, outage management
%
%\subsection{communication systems} 265
%ami, 
%
%\subsection{monitoring and diagnostics}
%
%\subsection{SMART METERS AND ADVANCED METERING INFRASTRUCTURE}
%
%vedere anche 3.4.1.4 



% http://ieeexplore.ieee.org/stamp/stamp.jsp?tp=&arnumber=6298960 

%http://www.smartgridinformation.info/pdf/5264_doc_1.pdf

% http://www.cs.nmsu.edu/~misra/papers/SmartGridSurvey.pdf


%libri 


\chapter{Sicurezza}
Il termine \textbf{``sicurezza"} si riferisce alle tecniche, ai processi e ai provvedimenti adottati per proteggere dati, reti di comunicazione, tecnologie informatiche e sistemi di calcolo da accessi non autorizzati o da attacchi. \newline
L'approccio tradizionale prevede che la maggior parte delle risorse a disposizione per mettere in sicurezza il sistema si focalizzi sulle componenti più cruciali e che le protegga dalle minacce più grandi e più note; questo meccanismo fa si che le componenti secondarie siano indifese e, inoltre, non protette da attacchi meno pericolosi. Tale approccio, però, risulta inefficiente nell'ambito della Smart Grid. \newline Per adattarsi al nuovo sistema, le organizzazioni promuovono un metodo più proattivo ed adattivo: il NIST, per esempio, ha recentemente pubblicato delle linee guida che consigliano uno spostamento verso il continuo monitoraggio e verso valutazioni real-time [ref].\newline \newline
La sicurezza della Smart Grid, in relazione al suo sviluppo, è un tema fortemente discusso: tutti concordano nel sostenere che la Smart Grid dovrebbe avere un modello di sicurezza robusto; il problema è che ci si trova dinanzi a due sfide: come poter rispondere ai requisiti richiesti e come poter applicare le numerose alternative esistenti quando si cerca di rendere sicuro un ambiente complesso come la Smart Grid.
\newline \newline
Quando si sente parlare di ``nuova tecnologia", di ``interconnessione" e di ``condivisione dei dati", subito ci si focalizza sui benefici e sulle nuove funzionalità che tali concetti portano con loro. C'è da considerare, però, anche i nuovi rischi che queste nuove funzionalità introducono all'interno del sistema.\newline Per questo motivo, lo scopo della sicurezza è quello di garantire che le funzionalità del sistema operino correttamente e siano protette da abusi. È importante sottolineare, però, che non esistono applicazioni, reti o sistemi completamente sicuri e le Smart Grid non sono un'eccezione. Sebbene ogni componente della nuova rete elettrica porti con se numerosi miglioramenti operazionali  o funzionali, introduce anche nuove vulnerabilità e rischi addizionali che, se non propriamente gestiti, possono portare il sistema ad essere esposto ad attacchi di varia natura.
\section{Cenni storici}

\section{Definire la sicurezza}
La sicurezza tradizionale fa affidamento sulla cosiddetta \textbf{CIA triad} [ref.], che ne costituisce il cuore. La CIA triad comprende tre concetti: confidentiality, integrity ed availability. \newline Una concezione più moderna, e più adatta all'ambiente della Smart Grid, prevede l'utilizzo del \textbf{Parkerian hexad}, proposto da Parker nel 2002 [ref.]. Tale modello propone, in aggiunta ai tre classici concetti precedenti, altri tre principi: control (o possession), authenticity ed usability (o utility). \newline All'interno di questi sei pilastri, è possibile trovare tutti i problemi relativi alla Smart Grid.
\begin{figure}[h]\centering{
  \includegraphics[scale=0.5, natwidth=674,natheight=679]{imgs/hexad.png}
  \caption{Parkerian hexad}
}
\end{figure}
\subsection{Confidentiality}
Tale concetto porta con sé una serie di problemi e di preoccupazioni relative alla trasmissione e alla memorizzazione di dati ricavati dalle operazioni della Smart Grid. Questo tipo di dati, infatti, è spesso ritenuto \textit{confidenziale}, nel senso che se fosse noto, avrebbe tutto il potenziale per causare danni alla sicurezza delle operazioni di tutto il sistema. \newline La confidenzialità, inoltre, può essere intesa anche in un'altra accezione: se i dati fossero noti alla concorrenza, per esempio,  quest'ultima potrebbe trarre un notevole vantaggio in uno specifico settore o in tutto il mercato. \newline A tali fattori si aggiungono altre nuove problematiche legate alla \textit{privacy del consumatore} e, quindi, dei suoi dati, che vengono fuori da meccanismi di metering quali l'AMI. Gli utenti, infatti, si aspettano che i consumi relativi alle loro abitazioni private rimangano confidenziali; se così non fosse, la disponiblità di tali informazioni insieme alla capacità di fare data mining, avrebbe il potenziale per creare significative preoccupazioni sulla privacy. \newline I punti della Smart Grid che introducono rischi per la confidenzialità, sono costituiti da tutte le locazioni in cui sono memorizzati i dati e da tutti i meccanismi di trasmissione delle informazioni. Per quanto riguarda i dati memorizzati, questi potrebbero essere letti, copiati e distribuiti a soggetti diversi dai destinatari. Per quanto riguarda la trasmissione, invece, sia su reti private che su reti pubbliche come Internet, i dati potrebbero essere intercettati, copiati e distribuiti. \newline La soluzione a tali problemi risiede nelle funzioni di \textit{cifratura dei dati} e di \textit{controllo degli accessi}. Fornendo l'appropriato livello di cifratura delle informazioni, quest'ultime possono essere protette da chiunque non sia il diretto destinatario. \newline Il controllo degli accessi, prevede che i dati siano protetti da coloro che hanno l'autorizzazione per accedere al sistema ma che, allo stesso tempo, non hanno bisogno di tali dati per svolgere il loro lavoro.

\subsection{Integrity}
L'integrità si riferisce all'abilità del sistema di evitare che le informazioni possano essere modificate da persone o da sistemi non autorizzati. \newline Se si rendono possibili meccanismi di modifica volontari quali la manipolazione dei dati, o anche involontari quali la loro corruzione, i sistemi riceveranno informazioni non accurate; a lungo andare ciò potrebbe avere un impatto negativo su tutte le operazioni e, in casi estremi, portare ad instabilità o compromettere del tutto la Smart Grid. \newline
I punti della nuova rete elettrica che introducono rischi per l'integrità, sono tutti quei punti che consentono il passaggio dei dati da un sistema ad un altro. Pertanto, la sicurezza di tali meccanismi di transizione è importante, ma ancora più importante è come il sistema che riceve i dati possa assicurarsi della validità di quest'ultimi: se i dati subiscono manipolazioni mentre sono in viaggio tra i due sistemi, il ricevente potrebbe prendere decisioni basate su tali informazioni (che risultano essere errate); se, invece, i dati sono soggetti a corruzione durante la loro transizione, ci si potrebbe trovare difronte ad un comportamento inaspettato del ricevente. In entrambi i casi, è evidente che l'integrità dei dati sia cruciale per assicurare la stabilità delle operazioni.   \newline
Le risposte ai problemi di integrità, possono essere trovate nei meccanismi di \textit{auditing}, di \textit{authorization}, di \textit{nonrepudiation}, e di \textit{message-signing}, che saranno trattati in seguito (vedi paragrafo 3.3, building blocks).


\subsection{Availability}
Molto spesso si tende ad utilizzare i concetti di reliability ed availability in maniera intercambiabile; in realtà, tali concetti hanno due significati diversi. La reliability, infatti, risponde alle seguenti domande: ``quanto spesso fallisce il sistema?", ``quanto è elastico?"; l'availability, invece, indica la disponibilità del sistema e, quindi, la capacità di compiere il lavoro che gli è stato assegnato, \textit{nel momento in cui se ne ha bisogno}. \newline Una porzione del sistema potrebbe essere attiva, eseguendo e processando i comandi, il 100\% del tempo, e pertanto molto affidabile ma, se le performance non sono adeguate ai bisogni della rete e operazioni critiche vengono ritardate o mancano, non si può dire che il sistema sia disponibile. \newline
I punti della Smart Grid che introducono rischi per la disponibilità sono troppi per poterli elencare: qualsiasi sistema, rete, dispositivo che gestisce le comunicazioni, processo per la gestione dei messaggi, e qualsiasi servizio invocato a qualsiasi livello applicativo e il suo sistema operativo sottostante sono un rischio per la disponibilità quando si trovano a dover gestire l'inoltro di un comando da un'estremità del sistema ad un'altra. Risolvere tale rischio è quasi tanto complicato quanto identificare le componenti del sistema che hanno il potenziale per impattare sulla disponibilità. La maggior parte delle soluzioni si affida a tecniche di ridondanza (clustering, bilanciamento del carico); il costo di tali metodologie, però, cresce in maniera proporzionale ai punti di fallimento che si identificano nel sistema. 

\subsection{Control}

\subsection{Authenticity}

\subsection{Usability}


\section{Building blocks}

\section{Minacce e loro impatto}

\section{Sforzo dello stato (?)}

\section{Compagnie}

\section{Servizi di terze parti}


\chapter{Standard e tecnologie}
L'infrastruttura di comunicazione consiste tipicamente di sistemi SCADA con canali di comunicazione dedicati da e verso il centro di controllo del sistema e di una Wide Area Network (WAN). I sistemi SCADA collegano tutte le principali strutture operative del sistema mentre la WAN è prevalentemente usata per azioni di mercato. Uno sviluppo importante per la Smart Grid (vedi Figura \ref{fig:cisg}) è quello di estendere la comunicazione a tutto il sistema di distribuzione e di stabilire una comunicazione bidirezionale con i clienti attraverso le Neighbourhood Area Network (NANs) che coprono le zone servite dalle sottostazioni di distribuzione. I clienti avranno la necessità di una Home Area Network (HAN) a cui saranno connessi gli smart device.
\begin{figure}[h]
	\centering
	\includegraphics[scale=0.310]{imgs/comm_inf_SG.png}
	\caption{Una possibile infrastruttura di comunicazione per la Smart Grid} \label{fig:cisg}
\end{figure}\\
Le sotto-reti di comunicazione che andranno a comporre la Smart Grid utilizzano diverse tecnologie (vedi Figura \ref{fig:th}) e di particolare interesse è il modo in cui quest'ultime possono essere integrate in maniera efficace. Le Smart Grid possono utilizzare diverse tecnologie di comunicazione wired e wireless (cellulare, satellitare, microwave, WiMAX etc.). Le tecnologie di comunicazione wireless short range, come WiFi e ZigBee,  sono tipicamente utilizzate nelle HAN.
In questo capitolo, saranno descritte alcune tecnologie di comunicazione associate ai livelli inferiori del modello di riferimento ISO/OSI.
\begin{figure}[h]
	\centering
	\includegraphics[scale=0.350]{imgs/tech.png}
	\caption{Tecnologie usate nelle differenti sottoreti} \label{fig:th}
\end{figure}

\section{Tecnologie di comunicazione}
\subsection{IEEE 802}
IEEE 802 è una famiglia di standard sviluppati per il supporto alle reti locali (LAN). Facendo riferimento alla Smart Grid, tali standard sono applicabili alle reti LAN in sistemi SCADA, NAN per le reti di distribuzione e HAN nei locali dei clienti. La Figura \ref{fig:arch_802} mostra come l'architettura IEEE 802 è incentrata sui due livelli inferiori del modello ISO/OSI.
\begin{figure}[h]
	\centering
	\includegraphics[scale=0.330]{imgs/arch_ieee802.png}
	\caption{Architettura IEEE 802} \label{fig:arch_802}
\end{figure}
Nella Figura in esame è mostrata la connessione di due LAN attraverso l'utilizzo di un Bridge. Tale connessione è comune in molte organizzazioni che hanno più LAN. Un pacchetto dalla sorgente va nel sottostrato Logical Link Control (LLC) che funge da interfaccia tra il livello di rete e il sottostrato MAC. LLC è definito da IEEE 802.2 e fornisce i meccanismi di controllo del flusso, multiplexing e di controllo degli errori. Il pacchetto passa poi nel sottostrato MAC in cui un header ed un trailer vengono aggiunti al pacchetto (a seconda della LAN cui il pacchetto entra). Poi si passa attraverso il livello fisico e nel canale di comunicazione e si raggiunge il Bridge. A livello MAC del Bridge, header e trailer vengono rimossi, recuperando così il pacchetto originale che passa al sottostrato LLC del Bridge. Successivamente il pacchetto viene elaborato dal sottostrato MAC (aggiungendo header e trailer appropriati) in base alla LAN a cui si trasmette. L'utilizzo del Bridge è essenziale in quanto LAN diverse utilizzano differenze dimensioni del frame e velocità (e.g. IEEE 802.3 utilizza un frame di 1500 byte, mentre IEEE 802.4 ne utilizza uno di 8191 byte\cite{802.3}).
\subsubsection{Ethernet}
Ethernet è diventata la tecnologia di rete più utilizzata per le LAN cablate grazie alla sua semplicità, affidabilità, facilità di manutenzione e la capacità di integrare nuove tecnologie. Essa ha un basso costo di installazione ed è facile farne l'upgrade. Si tratta di una tecnologia di comunicazione frame-based che si basa sullo standard IEEE 802.3. Ethernet utilizza un mezzo condiviso che può portare a collisioni tra i frame trasmessi dai vari host. Il problema delle collisioni è gestito da un protocollo chiamato Carrier Sense Multiple Access/Collision Detect (CSMA/CD). Un set di host connessi ad una rete in modo tale che la trasmissione simultanea da due host nel set porta a collisioni, crea un \emph{dominio di collisione}. Inoltre, le LAN Ethernet trasportano anche frame di broadcast il cui dominio raggiungibile è chiamato \emph{dominio di broadcast}. Le prestazioni della rete, in caso di traffico, sono influenzate dal modo in cui i domini di collisione e di broadcast sono posizionati e pertanto l'idea è quella di isolarli per aumentare le prestazioni della rete. La Figura \ref{fig:lan} mostra tali domini su una tipica LAN Ethernet.\newpage
\begin{figure}[h]
	\centering
	\includegraphics[scale=0.340]{imgs/lan.png}
	\caption{LAN Ethernet} \label{fig:lan}
\end{figure}
I Bridge limitano i domini di collisione mentre i Router limitano entrambi i domini. In Figura \ref{fig:lan} è mostrato come un pacchetto inviato dalla rete A può collidere con uno della rete B, ma non con uno inviato da C.
\subsubsection{Wireless}
IEEE 802.11 definisce un insieme di standard per le Wireless LAN (WLAN). L'interoperabilità dei dispositivi IEEE 802.11 è certificata dalla Wi-Fi Alliance. Una LAN Wireless è costituita dai seguenti componenti:
\begin{itemize}
	\item\emph{Station}: descrive qualsiasi dispositivo che comunica tramite una rete WLAN, ad esempio, un computer portatile, o cellulari che supportano WiFi. Nelle reti Ad-hoc questi dispositivi possono comunicare tra loro, creando una rete mesh (vedi Figura \ref{fig:bss}a). L'insieme di station che formano la rete Ad-hoc è chiamato Independent Basic Service Set (IBSS);
	\item\emph{Access Point (AP)}: consente ad una stazione di comunicare con un altra facendo da tramite. Necessita il doppio della larghezza di banda necessaria se la stessa comunicazione avvenisse direttamente tra le stazioni comunicanti. Gli AP rendono il sistema scalabile e consentono la connessione cablata con altre reti. In presenza di AP (vedi Figura \ref{fig:bss}b) l'insieme delle station è chiamato Infrastructure BSS;
	\item\emph{Distribution System}: interconnette Infrastructure BSS attraverso gli AP, come mostrato nella Figura \ref{fig:ds}. Facilita la comunicazione tra gli AP, l'inoltro del traffico da un BSS ad un altro ed il movimento di mobile station tra BSS. Un insieme di Infrastructure BSS è chiamato Extended Service Set (ESS).
\end{itemize}
La famiglia di reti LAN Wireless 802.11 utilizza il protocollo CSMA/CA per l'accesso al mezzo trasmissivo. Sono noti vari standard identificati da 802.11a/b/g/n/ac con variazioni a livello fisico. Una tipica applicazione di 802.11 nelle Smart Grid è mostrata in Figura \ref{fig:802_sg}.
\vspace{20pt}
\begin{figure}[h]
	\centering
	\includegraphics[scale=0.250]{imgs/bss.png}
	\caption{Architetture BSS di WLAN} \label{fig:bss}
\end{figure}
\vspace{30pt}
\begin{figure}[h]
	\centering
	\includegraphics[scale=0.450]{imgs/ds.png}
	\caption{Distribution System} \label{fig:ds}
\end{figure}
\begin{figure}[h]
	\centering
	\includegraphics[scale=0.350]{imgs/80211smartgrid.png}
	\caption{Applicazione di WLAN 802.11 in una Smart Grid} \label{fig:802_sg}
\end{figure}\newpage
\subsubsection{Bluetooth}
Bluetooth, definito dallo standard IEEE 802.15.1, è una tecnologia LAN wireless progettata per collegare i dispositivi mobili o fissi con bassi consumi, un corto raggio d'azione (fino a 100 metri di copertura) e un basso costo di produzione per i dispositivi compatibili. 
Bluetooth definisce due architetture di rete denominate Piconet e Scatternet. La Piconet è costituita da un dispositivo \emph{Master} e fino a sette dispositivi \emph{Slave}. Altri dispositivi possono sincronizzarsi col Master ma non possono partecipare alla comunicazione. Si dice che tali dispositivi sono in un parked state. Un device in parked state può passare in active state se il numero di Slave della Piconet è inferiore a sette. Le Piconet possono essere interconnesse attraverso un Bridge che può essere Slave per una Piconet e Master per un'altra oppure Slave per due Piconet che sono interconnesse come in Figura \ref{fig:bt}a e \ref{fig:bt}b. Un insieme di Piconet forma una Scatternet.\newpage
Per il trasferimento dei dati è possibile creare due tipi di collegamenti bluetooth:
\begin{itemize}
		\item Synchronous Connection Orientated (SCO) link
		\item Asynchronous Connectionless Link (ACL)
\end{itemize}
SCO è utilizzato quando la consegna tempestiva è più importante della consegna senza errori mentre ACL è utilizzato nel caso inverso.
\begin{figure}[h]
	\centering
	\includegraphics[scale=0.350]{imgs/bt.png}
	\caption{Piconet e Scatternet} \label{fig:bt}
\end{figure}
\subsubsection{ZigBee and 6LoWPAN}
ZigBee e 6LoWPAN sono due tecnologie di comunicazione basate su IEEE 802.15.4 per Wireless Personal Area Network (WPAN) dato il basso consumo, l'alta flessibilità ed i bassi costi. L'architettura protocollare di un device ZigBee è mostrata nella Figura \ref{fig:zbprot} in cui i due strati inferiori sono definiti da IEEE 802.15.4. Application Support e Network Layer per la rete ZigBee sono definiti dalla ZigBee Alliance\cite{zb}.
\begin{figure}[h]
	\centering
	\includegraphics[scale=0.400]{imgs/zbprot.png}
	\caption{Architettura Protocollare di ZigBee} \label{fig:zbprot}
\end{figure}\newpage
Un device ZigBee può essere un Full Function Device (FFD) o un Reduced Function Device (RFD). Una rete avrà almeno un FFD, che fungerà da coordinatore della WPAN. Il FFD può funzionare in tre modalità: coordinatore, router o device. Un RFD può funzionare solo come device. Un FFD può interagire sia con un altro FFD che con un RFD, mentre un RFD può parlare solo con un FFD.
La tecnologia ZigBee è considerata come una buona opzione per il metering e per la gestione dell'energia ideale in implementazioni Smart Grid data la semplicità, mobilità, robustezza e i bassi costi di sviluppo. Offre anche programmi di pricing e monitoraggio del sistema real-time. ZigBee presenta però alcuni vincoli relativi alle basse capacità di elaborazione, alla piccola dimensione della memoria e alle interferenze tra i vari apparecchi che condividono lo stesso mezzo trasmissivo. Tali problematiche, in condizioni di rumore, aumentano la possibilità di danneggiare il canale di comunicazione a causa delle interferenze. Schemi di interference detection/avoidance e protocolli di routing energy-efficient estendono il tempo di vita della rete e forniscono una performance di rete affidabile e ad alta efficienza dal punto di vista energetico.
\newline\newline
6LoWPAN è un protocollo che consente l'invio e la ricezione di pacchetti IPv6 nelle reti basate su IEEE 802.15.4. In tale protocollo è stato inserito un Adaptation Layer (vedi Figura \ref{fig:6pan}) per il collegamento tra lo strato MAC e il Network Layer IPv6.
%Il concetto 6LoWPAN nasce dall'idea che "il protocollo Internet potrebbe e dovrebbe essere applicato anche ai dispositivi più piccoli, e che i dispositivi a bassa potenza e con capacità di elaborazione limitate dovrebbero essere in grado di far parte dell' Internet of Things".
\begin{figure}[h]
	\centering
	\includegraphics[scale=0.400]{imgs/6pan.png}
	\caption{Architettura di rete 6LoWPAN} \label{fig:6pan}
\end{figure}
\\
Quando un RFD in una 6LoWPAN vuole inviare un pacchetto di dati ad un dispositivo che si trova al di fuori del dominio 6LoWPAN, invia inizialmente il pacchetto ad un FFD nella stessa WPAN. Il FFD che agisce da router, inoltrerà il pacchetto dati di hop in hop fino al gateway 6LoWPAN. Il gateway 6LoWPAN potrà quindi inoltrare il pacchetto al dispositivo di destinazione utilizzando l'indirizzo IP (vedi Figura \ref{fig:6pancom}).
\begin{figure}[h]
	\centering
	\includegraphics[scale=0.450]{imgs/6pancom.png}
	\caption{Comunicazione in una rete 6LoWPAN} \label{fig:6pancom}
\end{figure}
\subsubsection{WiMax}
Worldwide Interoperability for Microwave Access (WiMAX) è una tecnologia wireless conforme allo standard IEEE 802.16. Risulta superiore rispetto a Wi-Fi per velocità di trasmissione e range di copertura delle celle, per cui è adatto ad una trasmissione sia di tipo urbano che rurale. Inoltre, implementa diverse tecniche di crittografia, sicurezza ed autenticazione. WiMAX è una tecnologia in grado di integrarsi con quelle presenti, soddisfacendo diverse specifiche imposte da una tipica Smart Grid tra cui la massima accessibilità ed interoperabilità, tempi di latenza inferiori ai 50ms e larghezza di banda di 5MHz. Fornisce sia connettività fissa che mobile usando una tecnica chiamata Orthogonal Frequency Division Multiple Access (OFDMA). Una tipica rete WiMax è mostrata in Figura \ref{fig:wim}. La copertura di WiMax si estende fino ai 50 km con una velocità di trasmissione dati pari a 75 Mbps per i collegamenti fissi e fino a 15 Mbps per le connessioni mobili. È ottimizzato per supportare dispositivi mobili fino ad una velocità di 10 km/h. Anche se supporta veicoli in movimento fino a 120 km/h, le sue prestazioni degradano con l'aumentare della velocità del veicolo\cite{wimax}.
\begin{figure}[h]
	\centering
	\includegraphics[scale=0.350]{imgs/wim.png}
	\caption{Una rete WiMax} \label{fig:wim}
\end{figure}
\newpage
\subsection{Power line}
La Power Line Communication (PLC) rappresenta una delle tecnologie di rete proposte per la trasmissione in ambiente Smart Grid in quanto l'infrastruttura esistente ne riduce i costi di installazione. Se, da un lato non è richiesta la realizzazione di nuove strutture, da un altro lato vi è un limite dovuto alla presenza di disturbi che possono corrompere le informazioni, non garantendo più la continuità del servizio. PLC trasporta i dati utilizzando i conduttori e le linee elettriche esistenti. Fornisce servizi di comunicazione per Automatic Meter Reading (AMR), AMI e HAN ma anche l'accesso ad internet all'utente finale. In una tipica rete PLC, gli smart meter sono collegati al data concentrator (che colleziona le informazioni ricevute dai vari meter) attraverso power line e i dati vengono trasferiti al data center tramite tecnologie di rete cellulare. La tecnologia PLC è infatti scelta per la comunicazione tra gli smart meter e il data concentrator, mentre la tecnologia GPRS è utilizzata per trasferire i dati dal concentrator al data center.\newline\newline
L'ENEL, nota azienda multinazionale produttrice e distributrice di energia elettrica, ha scelto la tecnologia PLC per trasferire i dati degli smart meter al data concentrator più vicino e la tecnologia GSM per inviare i dati al data center.
La topologia di rete, il numero/tipo dei dispositivi collegati e la distanza trasmettitore/ricevitore compromettono la qualità del segnale.
\newpage
Le sensibilità di PLC ai disturbi e alla qualità del segnale sono gli svantaggi che rendono la tecnologia non adatta alla trasmissione dei dati. Tuttavia, ci sono state alcune soluzioni ibride in cui la tecnologia PLC si combina con altre, ad esempio, GPRS o GSM, per fornire una connettività non possibile generalmente con PLC.\newline\newline
Inizialmente, la velocità di trasmissione in questo tipo di reti era molto limitata, fino a pochi kbps. Successivamente, grazie al progresso tecnologico e con l'introduzione di broadband PLC (BB-PLC), un'applicazione della tecnologia PLC a banda larga che fornisce l'accesso a Internet tramite linee elettriche ordinarie, la velocità di trasmissione ha raggiunto anche i 200 Mbps. Sono utilizzate tre tecnologie di comunicazione che prendono il nome di \emph{narrowband transmission}, \emph{spread-spectrum transmission} e \emph{DSP-processed narrowband transmission}. La Figura \ref{fig:vs_plc} mostra alcuni vantaggi e svantaggi relativi a PLC in ambito Smart Grid. Tra gli standard e i protocolli maggiormente utilizzati troviamo IEEE P1901 e HomePlug.\vspace{20pt}
\begin{figure}[h]
	\centering
	\includegraphics[scale=0.500]{imgs/vs_plc.png}
	\caption{Vantaggi e Svantaggi di PLC in ambito Smart Grid} \label{fig:vs_plc}
\end{figure}
\newpage
\subsubsection{IEEE P1901}
Il gruppo IEEE P1901 è stato formato nel 2005 con lo scopo di sviluppare una tecnologia per la trasmissione di voce o dati che utilizzasse la rete di alimentazione elettrica come mezzo trasmissivo. Lo standard permette una comunicazione ad altà velocità tra i device che prendono il nome di BPL (Broadband over Power Line). Lo standard utilizza frequenze inferiori a 100 MHz ed è di supporto ai device BPL utilizzati per i collegamenti first-mile/last-mile così come quelli utilizzati nelle reti LAN all'interno di edifici. Inoltre, tali device possono essere utilizzati all'interno di smart energy application, autoveicoli e in altre applicazioni per la distribuzione dei dati.
\subsubsection{HomePlug}
HomePlug è una tecnologia broadband non standardizzata e specificata dalla HomePlug Powerline Alliance, i cui membri sono le principali aziende nel settore della comunicazione e dell'energia. Il protocollo gestisce vari sottocanali suddividendo la larghezza di banda disponibile. La velocità di trasmissione varia da 1 a 14 Mbps e i nodi sono in grado di adattarsi al data rate ottimale in maniera automatica. Le collisioni sono evitate grazie al CSMA/CD. Lo standard HomePlug 1.0 per la connessione di dispositivi nelle case (1-10 Mbps) fa uso della tecnica di Orthogonal Frequency Division (OFDM), utilizzata anche da DSL, IEEE 802.11a e IEEE 802.11g. Il rumore, comune in ambiente power line, è superato per mezzo di forward error correction e data interleaving. La HomePlug Powerline Alliance ha definito ulteriori standard come HomePlug AV/AV2 che forniscono banda sufficiente per applicazioni come HDTV e VoIP, HomePlug CC e HomePlug BPL.
\section{Standard per lo scambio di informazioni}
\subsection{Standard per Smart Meter}
Gli smart meter possono essere utilizzati in vari modi portando a differenti requisiti dal punto di vista del sistema di comunicazione. Con Automated Meter Reading (AMR) si richiede una trasmissione occasionale dei dati energetici registrati (circa una volta al mese), viceversa con  Advanced Metering Infrastructure (AMI) si richiedono frequenti comunicazioni bidirezionali (ad esempio ogni 30 minuti). ISO/IEC 62056 e ANSI C12.22 sono due famiglie di standard che descrivono sistemi di comunicazione per gli smart meter. ISO/IEC 62056 definisce Transport e Application Layer per lo smart metering nell'ambito di una serie di specifiche chiamate COSEM (Companion Specification for Energy Metering). ANSI C12.22 (vedi Figura \ref{fig:arch_c1222}) specifica l'invio e la ricezione dei dati registrati da e verso sistemi esterni ed è possibile utilizzarlo su qualsiasi rete di comunicazione.
\begin{figure}[h]
	\centering
	\includegraphics[scale=0.400]{imgs/arch_c1222.png}
	\caption{Architettura ANSI C12.22} \label{fig:arch_c1222}
\end{figure}
\subsection{Modbus}
Modbus è un protocollo di messaggistica che risiede nell'Application Layer e consente la comunicazione tra i dispositivi collegati su diversi bus e reti. Può essere implementato tramite Ethernet o utilizzando la trasmissione seriale asincrona su EIA 232, EIA 422, EIA 485 e fibra ottica. Di questi, l'applicazione più comune è Modbus su EIA485. La Figura \ref{fig:modbus} mostra come l'Application Layer Modbus è connesso agli altri layer del modello OSI.
Modbus su EIA 485 è ampiamente utilizzato nell'automazione delle sottostazioni. La comunicazione è avviata dal Master con una query. Il Master è l'unico che può inviare query destinate al singolo Slave o di broadcast. Uno Slave monitora continuamente la rete riconoscendo solo le query destinate ad esso. All'arrivo di una query, lo Slave eseguirà un'azione o risponderà. Tra i problemi del protocollo spiccano il limitato supporto alle varie tipologie di dati e la non garanzia di sicurezza.
\begin{figure}[h]
	\centering
	\includegraphics[scale=0.500]{imgs/modbus.png}
	\caption{Modbus stack} \label{fig:modbus}
\end{figure}
\newpage
\subsection{DNP3}
DNP3 è un insieme di protocolli di comunicazione utilizzato tra i componenti nei sistemi di automazione. DNP3 gioca un ruolo fondamentale all'interno dei sistemi SCADA, utilizzato dalle SCADA Master Station (conosciute anche come \emph{centri di controllo}) per comunicare con le Remote Terminal Unit (RTU) e/o gli Intelligent Electronic Device (IED). DNP3 è stato recentemente adottato come standard IEEE 1815-2010\cite{dnp}. Lo User Layer DNP prende input analogici/binari e da in output segnali analogici/binari. Un Master DNP3 invia richieste e le stazioni Slave DNP3 rispondono. Uno Slave DNP3 può anche trasmettere un messaggio senza aver ricevuto alcuna richiesta. Il Physical Layer DNP3 utilizza alcuni tra i più noti protocolli di comunicazione seriali come EIA 232 o EIA 485.
\newline\newline
Poiché in applicazioni Smart Grid generalmente si presume l'accesso di terze parti alla stessa rete e all'infrastruttura sottostante basata su protocollo IP, un passo in avanti è stato quello che ha portato all'aggiunta di un'autenticazione sicura al protocollo DNP3. Alcuni vendor supportano la crittografia via \textbf{bump-in-the-wire} (cioè con dispositivi esterni che cifrano e decifrano i dati agli estremi del canale di comunicazione) per le comunicazioni seriali e tramite VPN per le reti IP. Uno dei metodi bump-in-the-wire più popolari è noto come \emph{AGA-12} (American Gas Association).
\subsection{ISO/IEC 61850 \label{sec:IEC61850}}
ISO/IEC 61850 è uno standard per la progettazione dei sistemi di automazione per le sottostazioni elettriche. E' una sovrastruttura che coordina e gestisce protocolli e tecnologie esistenti garantendo l'interoperabilità. Generalmente, questi protocolli girano su reti TCP/IP o LAN con switch Ethernet molto performanti per rispondere ai requisiti stringenti dei dispositivi, che necessitano di tempi di risposta inferiori a 4-5 millisecondi.\newline\newline
I principali vantaggi dello standard ISO/IEC 61850:
\begin{itemize}
	\item Coordina la complessità di tante unità indipendenti;
	\item Si integra con i sistemi preinstallati in rete;
	\item E' scalabile e facilita l'integrazione di apparati diversi;
	\item Si basa il più possibile su standard esistenti;
	\item E' aperto e supporta i \emph{self descriptive device} eliminando problemi di configurazione manuale;
	\item Si basa sui \emph{data object} e standardizzazione degli elementi tipici di una rete elettrica;
	\item Permette di ottenere alte prestazioni di multicast;
	\item E' estensibile e flessibile in modo da adattarsi rapidamente alla configurazione del sistema.
\end{itemize}
\newpage
La struttura dello standard ISO/IEC 61850\cite{iec61850} è mostrata in Figura \ref{fig:iec61850}.
\begin{figure}[h]
	\centering
	\includegraphics[scale=0.400]{imgs/iec61850.png}
	\caption{Struttura dello standard ISO/IEC 61850} \label{fig:iec61850}
\end{figure}\\
ISO/IEC 61850 suddivide ogni sottostazione in tre livelli\cite{iec61850} chiamati \emph{Station Level}, \emph{Bay Level} e \emph{Process Level}. La suddivisione dei livelli è mostrata in Figura \ref{fig:61850ls}.

\begin{figure}[h]
	\centering
	\includegraphics[scale=0.400]{imgs/61850ls.png}
	\caption{Livelli di una sottostazione} \label{fig:61850ls}
\end{figure}

Il protocollo identifica le funzioni e le caratteristiche dei dispositivi fisici che si modellano in uno o più dispositivi logici. I dispositivi logici sono a loro volta suddivisi in nodi logici che sono in relazione tra loro in base a \emph{data} e \emph{data attribute} (vedi Figura \ref{fig:iec61850ln}).
\begin{figure}[h]
	\centering
	\includegraphics[scale=0.350]{imgs/iec61850ln.png}
	\caption{Device Model ISO/IEC 61850} \label{fig:iec61850ln}
\end{figure}
\newpage
Come mostrato in Figura \ref{fig:iec61850ds}a, un device model considera inizialmente un physical device. Tale modello consente ad un singolo dispositivo fisico di agire da gateway di informazioni per più dispositivi. Successivamente vengono specificati i logical device all'interno di tale dispositivo. Ogni logical device contiene uno o più logical node, logicamente correlati ad una funzione della stazione. I logical node sono definiti da gruppi di data object e relativi servizi, ognuno modellato secondo gli schemi definiti dalle \textbf{Common Data Classes} (CDC).\newline
\begin{figure}[h]
	\centering
	\includegraphics[scale=0.350]{imgs/iec61850ds.png}
	\caption{ISO/IEC 61850 data structure} \label{fig:iec61850ds}
\end{figure}\newpage
La Figura \ref{fig:name_obj} mostra un esempio di nome per un oggetto in un formato standard. Utilizzando tale formato si è in grado di indicare le informazioni relative allo status o alla posizione di un dispositivo. I logical node sono identificati con nomi definiti dallo standard in cui la prima lettera indica l'attinenza (e.g. A controllo automatico, M misura, X switchgear, etc).
\begin{figure}[h]
	\centering
	\includegraphics[scale=0.350]{imgs/name_obj.png}
	\caption{Struttura del nome di un oggetto} \label{fig:name_obj}
\end{figure}
\newline
L'ISO/IEC 61850 si basa sulla specifica degli \emph{oggetti} e dei \emph{servizi astratti} di comunicazione che permettono di scrivere un'applicazione indipendentemente dai protocolli tradizionali. Si definisce quindi un modello che prende il nome di Abstract Communication Service Interface (ACSI) che definisce l'insieme dei servizi e le risposte a quei servizi che rendono gli IED uguali dal punto di vista della rete. Inoltre, tale modello interpreta i dati e gli attributi dei vari elementi garantendone l'interoperabilità. Gli oggetti e i servizi definiti dall'ACSI vengono implementati attraverso il protocollo ISO-9560 Manufacturing Message Specification (MMS). Si tratta di un protocollo flessibile in grado di supportare funzioni complesse e la logica a oggetti ACSI. Tale protocollo definisce i messaggi di comunicazione tra i vari centri di controllo oppure tra le stazioni e i centri di controllo.\newline\newline
Il concetto di astrazione e standardizzazione presuppone l'utilizzo di un linguaggio comune di configurazione. L'ISO/IEC 61850 si serve di un linguaggio basato su \emph{XML} chiamato \emph{Substation Configuration Language} (SCL). Grazie all'utilizzo di un linguaggio standard è possibile garantire l'interoperabilità tra gli IED che usano diversi protocolli, permettere una configurazione automatica dei dispositivi, ridurre la presenza di errori dovuti all'intervento umano nella gestione degli IED e consentire maggiore trasportabilità facendo in modo che ogni IED, che supporta ISO/IEC 61850, presenti un file SCL che ne definisce la configurazione. Nella Figura \ref{fig:iec61850ds}b è mostrato un esempio di IED.\newpage
Gli elementi fondamentali del modello di una sottostazione sono elencati di seguito (vedi Figura \ref{fig:arch_iec61850}):
\begin{itemize}
	\item\textbf{Merge Unit}: a livello di processo, i dati raccolti da sensori ottici, elettronici, da TV e TA, sui valori di tensione e corrente o sullo status dei componenti, sono recuperati dalle Merge Unit (MU). In genere, sono situate in corrispondenza del centro di controllo;
	\item\textbf{Intelligent Electronic Device}: gli IED che supportano ISO/IEC 61850 comunicano attraverso le MU e un process bus Ethernet a 10 Gbps;
	\item\textbf{Wrapper ISO/IEC 61850}: gli IED che non supportano il protocollo ISO/IEC 61850 utilizzano un Wrapper;
	\item\textbf{Process Bus e Station Bus}: le MU comunicano con il Bay Level attraverso un process bus mentre tutti i nodi logici (IED) comunicano attraverso un bus Ethernet a 100 Mbps;
	\item\textbf{Gateway e Internet}: diverse sottostazioni comunicano tra loro attraverso la rete Internet. Un Gateway permette di collegarsi alla rete e accedere alle informazioni da un centro di controllo o da remoto seguendo una procedura che garantisce la robustezza e sicurezza del sistema.
\end{itemize}

\begin{figure}[h]
	\centering
	\includegraphics[scale=0.450]{imgs/arch_iec61850.png}
	\caption{Architettura del sistema con lo standard ISO/IEC 61850} \label{fig:arch_iec61850}
\end{figure}
\newpage
Lo standard ISO/IEC 61850 si serve dei seguenti strumenti per la gestione delle informazioni (vedi Figura \ref{fig:tools_iec61850}):
\begin{itemize}
	\item Generic Substation Event (GSE);
	\begin{itemize}
		\item Generic Object Oriented Substation Event (GOOSE);
		\item Generic Substation State Event (GSSE).
	\end{itemize}
	\item Sampled Measured Values (SMV);
	\item Time Synchronization;
	\item Report e Logging.
\end{itemize}
\begin{figure}[h]
	\centering
	\includegraphics[scale=0.350]{imgs/tools_iec61850.png}
	\caption{Tools di ISO/IEC 618504} \label{fig:tools_iec61850}
\end{figure}
\textbf{GSE} è un protocollo che fornisce uno strumento veloce ed affidabile per la segnalazione di \emph{eventi} all'interno della sottostazione. Le caratteristiche principali sono il servizio multicast/broadcast con modello di comunicazione publish/subscribe. I messaggi sono trasmessi in formato binario. GSE prevede due modelli di servizio che prendono il nome di GOOSE e GSSE.\newline\newline
\textbf{GOOSE} è stato progettato per essere \emph{brand indipendent}. Utilizza Virtual LAN, stabilendo più network virtuali sulla stessa rete fisica e determinando livelli di priorità per i messaggi, consentendone anche la ritrasmissione (un identificativo indica se il messaggio è nuovo o ritrasmesso).\newpage
\textbf{GSSE} è utilizzato per lo scambio di informazioni sui soli cambiamenti di stato. In questo caso i messaggi sono costituiti da una serie di bit che rappresentano liste di stati. GSSE necessita un tempo di trasmissione maggiore se confrontato con GOOSE.
\newline\newline
\textbf{SMV} è un protocollo per lo scambio di dati e la trasmissione di misure prodotte dai trasduttori delle sottostazioni: permette lo scambio di segnali tra gli IED. Prevede due metodi di comunicazione:
\begin{itemize}
	\item\emph{SV over Serial Unidirectional Multidrop Point-to-Point fixed link}: Sistema di comunicazione unidirezionale che specifica una serie di dataset (tensioni trifase, correnti trifase, etc). I valori analogici vengono codificati a 16 bit;
	\item\emph{SV over Ethernet}: Versione più generica e flessibile di SV che fornisce la possibilità di definire dataset con valori di diversa dimensione e tipo, configurabili dall'utente grazie all'utilizzo di SCL. Utilizza un modello di comunicazione publish/subscribe con possibilità di multi-casting.
\end{itemize}
\textbf{Time Synchronization} è un servizio di sincronizzazione dei clock fondamentale per applicazioni real-time. Utilizza un subset di Network Time Protocol (NTP) con riferimento allo Universal Coordinated Time (UTC). NTP è il protocollo tipicamente utilizzato per sincronizzare i clock di computer collegati ad Internet e in LAN.\newline\newline
\textbf{Report} è lo strumento che permette di memorizzare i cambiamenti dei dati e degli attributi relativi ai nodi logici. Genera dataset contenenti attributi di interesse e richiede ai nodi logici l'invio delle informazioni riguardanti le variazioni nel sistema.\newline\newline
\textbf{Log} è la registrazione degli eventi relativi ad un dispositivo. I log sono registrati in un server e, a differenza dei Report, i dispositivi logici creano al loro interno un database di eventi senza inviarne notifica.\newline\newline
Tra i limiti di ISO/IEC 61850 si evidenziano i costi elevati per l'installazione dei server e dei dispositivi atti alla gestione dei dati ma anche  complessità dal punto di vista dell'architettura. ISO/IEC 61850 è affiancato da ISO/IEC 62351 che garantisce la sicurezza e specifica i requisiti tecnici che devono essere rispettati dai fornitori.
\newpage
\section{Standard per la sicurezza}
Gli standard per la sicurezza informatica sono di recente invenzione e fondamentali data la grande mole di informazioni sensibili memorizzate sui computer che sono collegati ad Internet. Inoltre, molte attività che prima erano condotte manualmente, oggi sono svolte dalle macchine in maniera automatica introducendo quindi un maggiore bisogno di affidabilità e di sicurezza in tali sistemi informatici. Come ampiamente discusso nel Capitolo 4, la sicurezza è un fattore importante per per gli individui che devono proteggersi dal cosiddetto furto di identità ma anche per le aziende perché devono proteggere i loro segreti industriali e le informazioni sui dati personali dei clienti.\newline\newline
I problemi relativi alla sicurezza vanno quindi dagli accessi non autorizzati a informazioni recuperate dagli smart meter, lo spegnimento dei dispositivi da parte di un attaccante così come un attacco alla Smart Grid per causare un'interruzione al passaggio di corrente. I problemi relativi alla privacy riguardano invece l’alta frequenza con cui vengono effettuate le letture per misurare il consumo energetico in quanto esse mostrano totalmente il comportamento dell’utente. In generale si cerca di aggregare le informazioni dei meter per rilevare sia frode che perdite (per esempio nel caso del gas, dove un'eventuale perdita pone un problema di sicurezza) utilizzando schemi di aggregazione \emph{privacy-friendly} in modo da mascherare i singoli consumi dei meter.
\newline\newline
Nell'ambito dei power system, esistono diversi standard che si applicano alla sicurezza delle apparecchiature all'interno delle sottostazioni e molti sono in fase di sviluppo. Per la valutazione complessiva della sicurezza, è ampiamente utilizzata la norma ISO 27001 e specifica la valutazione dei rischi e la strategia da utilizzare per lo sviluppo di un sistema di sicurezza in modo da limitarli.
\subsection{ISO/IEC 62351}
ISO/IEC 62351 è uno standard sviluppato dal WG15 facente parte della TC57 dell'organo internazionale IEC. Questo standard è stato sviluppato per gestire la sicurezza nella serie di protocolli della commissione tecnica 57, tra i quali le serie ISO/IEC 60870-5, ISO/IEC 60870-6 series, ISO/IEC 61850, ISO/IEC 61970 e ISO/IEC 61968.\newpage
Tra i diversi obiettivi di sicurezza che lo standard persegue ci sono:
\begin{itemize}
	\item Autenticazione nel processo di trasferimento di dati tramite firma digitale;
	\item Garanzia di accessi esclusivamente dopo autenticazione;
	\item Prevenzione dell'\emph{eavesdropping} (ossia intercettazioni della comunicazione non autorizzate);
	\item Prevenzione da attacchi di \emph{playback} e attacchi di  \emph{spoofing} (ovvero sostituirsi ad una controparte della comunicazione);
	\item Rilevamento delle intrusioni.
\end{itemize}
Le prime due parti di ISO/IEC 62351 comprendono la spiegazione di differenti scenari e la definizione di alcuni termini. ISO/IEC 62351 utilizza i protocolli di sicurezza esistenti, come il Transport Layer Security (TLS, \cite{tls}), utilizzato con successo in altre aree tecniche e in diverse parti dello standard. TLS offre servizi di sicurezza come la mutua autenticazione dei peer in una comunicazione (utile per contrastare attacchi  \emph{man-in-the-middle}) ma si occupa anche dell'integrità e della riservatezza dei dati comunicati. La terza parte di ISO/IEC 62351 definisce come possono essere forniti servizi di sicurezza in una  comunicazione TCP/IP based e specifica suite di cifratura. La quarta parte specifica le procedure, i miglioramenti del protocollo, e gli algoritmi atti a promuovere l'aumento dei messaggi di sicurezza trasmessi su MMS. Questa parte definisce alcune procedure relative a Transport e Application Layer, basate su TLS, in modo da proteggere le informazioni trasmesse. La quinta parte riguarda invece la comunicazione seriale in cui si vanno a definire ulteriori misure di sicurezza per proteggere l'integrità delle connessioni seriali applicando chiavi hash. Questa parte specifica quindi anche una separata gestione delle chiavi. La sesta parte di ISO/IEC 62351 si occupa dei profili di ISO/IEC 61850 per la comunicazione, non basati su TCP/IP. La settima parte è utile poichè implementa e/o estende sistemi di intrusion detection e l'ottava ed ultima parte si occupa di role-based access control.
\chapter{Analisi delle problematiche}
% !TeX spellcheck = it_IT
Negli ultimi anni, un numero sempre crescente di infrastrutture critiche è stato digitalizzato, aggiungendo capacità di comunicazione e di computazione a numerosi dispositivi nelle reti di distribuzione dell'energia e dell'acqua, sistemi di trasporto, e manifatturieri. Ciò avviene in parte con lo scopo di aumentare l'efficienza, ma spesso è anche un requisito necessario a gestire l'ambiente che muta, come ad esempio la generazione locale dell'energia o il passaggio ai veicoli elettrici nel caso della distribuzione energetica.\\
Un progetto di digitalizzazione molto visibile è il passaggio dai \emph{meter} analogici ai digitali (\emph{smart}), che è attualmente in corso in vari paesi del mondo. Oltre ad una fatturazione complessivamente più precisa, uno smart meter può anche dare input agli algoritmi di controllo della grid, essere usato nei mercati energetici, comunicare con la \emph{smart home} (ad esempio, per regolare l'aria condizionata ed i sistemi di riscaldamento quando la richiesta energetica è alta), oppure per disconnettere da remoto un consumatore. In questo modo, un dispositivo precedentemente disconnesso e non critico si trasforma in un dispositivo connesso che può generare dati \emph{process-critical}.\\
La robustezza dei dati e dei comandi di switch è vitale - se una grande quantità di famiglie viene disconnessa simultaneamente, l'energia in eccesso non ha dove andare, potrebbe danneggiare la grid. In maniera simile, se gli algoritmi di manutenzione si basano su dati provenienti da misure effettuate dagli smart meter, input errati possono produrre effetti di gran lunga peggiori delle frodi di fatturazione. Ciò pone una nuova sfida per i produttori di meter: progettare dispositivi economici, largamente distribuiti e che lavorino su canali con banda banda molto ristretta.
\section{Open Smart Grid Protocol}
L'Open Smart Grid Protocol (OSGP) è stato uno dei primi protocolli di comunicazione su powerline per smart meters disponibile sul mercato, ed è largamente utilizzato per comunicare tra smart meter e l'aggregatore di dati, il quale è un dispositivo che colleziona dati provenienti da diverse centinaia di meter in un segmento di PLC (Power Line Communication) e li inoltra ad un controllore centralizzato.\\
L'OSGP risiede nel layer delle applicazioni dello stack protocollare definito dallo standard 14908 ISO/IEC \cite{standard14908}. Nonostante l'OSGP sia principalmente utilizzato per applicazioni di smart-metering, esso è stato progettato per un utilizzato più ampio all'interno di dispositivi della Smart Grid. Lo stack protocollare è molto leggero. Tale leggerezza è ottenuta pagando in sicurezza, infatti le primitive crittografiche consigliate dal NIST  (ad esempio: Advanced Encryption Standard - AES, in \emph{authenticated mode}) sono evitate, optando per altre meno intense computazionalmente: lo stream cipher RC4 per la cifratura ed una funzione digest non standard per l'autenticazione dei messaggi.\\
Il problema più grande è la combinazione di uno stream cipher con una funzione di digest lineare, che apre le possibilità per attacchi alle chiavi crittografiche ed ai messaggi. Ad esempio, si potrebbe sfruttare tale vulnerabilità utilizzando il ricevente di un messaggio come un ``oracolo". Dato un messaggio propriamente cifrato ed autenticato, un attaccante ipotizza una combinazione di un bit del messaggio ed un bit della chiave, invia un messaggio al ricevente appositamente confezionato, ed utilizza la risposta del  dispositivo che rifiuti o confermi l'ipotesi. Siccome questo può essere effettuato per singoli bit, un attaccante può ricostruire la chiave principale del dispositivo (OMA key) con al più $96 \times 3$ risposte.
\subsection{Potenziali debolezze del protocollo}
Sono state identificate quattro potenziali debolezze nel protocollo OSG.
\subsubsection{Utilizzo di RC4}
Negli ultimi anni sono state identificate varie debolezze nello stream cipher RC4. Uno dei problemi è la correlazione tra la chiave ed il \emph{keystream} di RC4 che può essere utilizzata per ricostruire la chiave. Questa debolezza è stata sfruttata con successo attraverso i vettori di inizializzazione (IVs)  utilizzati per ``rompere" WEP (Wired Equivalent Protection), standard che fa largo uso di RC4. Gli attacchi \cite{wep1}, \cite{wep2} rompono WEP in pochi secondi e tool per l'attacco come Aircrack-ng \cite{aircrackng} sono disponibili gratuitamente per effettuare penetration testing \cite{kali}.\\
L'uso di RC4 nel protocollo OSG è simile al modo in cui RC4 era usato nello standard WEP. Mentre, per ogni messaggio, è generata una nuova chiave, solamente i primi 8 byte della chiave variano, ed i restanti 8 byte sono costanti. Inoltre, gli 8 byte iniziali vengono messi in XOR con un valore noto pubblicamente. Sebbene gli autori non siano a conoscenza di un attacco reale a tale caratteristica, ciò implica una forte correlazione tra le chiavi usate, e fornisce ad un attaccante dati interessanti da analizzare. Ulteriormente, l'impostazione è simile a quella utilizzata nello standard WEP, questo lascia intuire che c'è una grande esperienza nel campo dello \emph{statistical key recovery}.\\
Un altro problema di RC4 in OSGP è che è utilizzato con una autenticazione debole. Data la natura di uno stream cipher, un attaccante che entra in possesso di un \emph{ciphertext} può alterare bit in posizioni scelte. Grazie alla conoscenza dello spazio dei messaggi dello standard EN 14908, un attaccante può inviare messaggi alterati che sono validi con alta probabilità. Inoltre, egli può autenticare questo messaggi sfruttando la debolezza della funzione di digest descritta qui di seguito.
\subsubsection{Debole funzione di digest}
La funzione di digest utilizzata nel protocollo OSG per l'autenticazione dei messaggi è lineare ed è implementata in un modo che non solo disabilita l'autenticazione ma può inoltre decifrare un messaggio intercettato in tempo lineare nella lunghezza del messaggio. La debolezza strutturale intrinseca alla funzione di digest consente un attacco ``man-in-the-middle" (che dimezza la taglia della chiave per un attacco di forza bruta), la modifica dei messaggi in maniera controllata, ricomputazione del digest corretto senza il bisogno di cifratura - o di chiavi di autenticazione, ed utilizzando i messaggi di NACK di un dispositivo su un digest errato per ricostruire sia la chiave di autenticazione che il messaggio.
\subsubsection{Sicurezza di broadcast non definita}
Nonostante la funzione di broadcast sia chiamata ``Secure Broadcast", la sicurezza definita su di essa è molto bassa. Questo è particolarmente preoccupante, se si pensa che tale meccanismo è utilizzato per inviare aggiornamenti del firmware, che sono distribuiti attraverso il meccanismo di broadcast sicuro senza alcuna misura specifica atta a fornire autenticazione dalla sorgente.
\subsubsection{Utilizzo della chiave}
Sebbene il protocollo faccia uso di chiavi di sessione per la cifratura, utilizza una master key per l'autenticazione. Comunque, questa chiave di autenticazione è usata per derivare la cifratura delle chiavi di sessione. Quindi se la chiave di autenticazione è compromessa, tutte le chiavi di sessione sono note dall'attaccante. Siccome la chiave di autenticazione è usata con un debole algoritmo di autenticazione, essa è altamente esposta ad un gran numero di attacchi, e comprometterla è ben possibile.
\subsection{Sfruttare le debolezze strutturali}
Queste debolezze strutturali possono essere sfruttate in vari modi. Gli attacchi adatti a tale scopo complessivamente non sono molto sofisticati, piuttosto, grazie alla debolezza dei \emph{building block} crittografici un attaccante è capace di dirottare un intero segmento PLC.\\
Un attacco relativamente semplice punta a \textit{rompere} la funzione di digest così che un attaccante possa non solo forgiare messaggi, ma - avendo ottenuto accesso ad un messaggio autentico di sufficiente lunghezza - anche ricostruire chiavi crittografiche con basso sforzo, per poi inviare falsi comandi correttamente autenticati.
%-----------------------------------------------------------------------------
%\section{Attacchi}
\section{Attacking Smart Meters and Smart Devices}
Uno dei maggiori argomenti utilizzati per mettere in sicurezza gli smart meter è che i consumatori avranno accesso fisico, e potenzialmente anche logico, a tali dispositivi. Sebbene i consumatori avessero accesso ai vecchi meter, questi ultimi non operavano utilizzando tecnologie familiari ai consumatori o a cui potessero avere accesso. A causa della prevalenza di queste tecnologie ben note, gli smart meter possono essere trattati come obiettivi tradizionali, quando ci si trova ad applicare metodologie di \emph{security testing} su di essi.\\
Di seguito vengono analizzate due tra le più comuni metodologie di security testing e come queste possano essere applicate al testing degli smart meter.

\subsection{Open Source Security Testing Methodology Manual}
Nel Gennaio 2001 , in USA e Spagna, fu fondato l'Institute for Security and Open Methodologies (ISECOM): organizzazione no-profit il cui scopo è quello di fornire soluzioni pratiche per security awareness, ricerca, certificazione e business integrity. Il loro \emph{Open Source Security Testing Methodology Manual} (OSSTMM) fornisce agli utenti le metodologie per effettuare security testing. L'OSSTMM contiene sei sezioni che recensiscono un numero enorme di aspetti di sicurezza, incluse reti, dispositivi wireless, e sicurezza fisica. Per tali aspetti, è possibile applicare l'OSSTMM al security testing degli smart meter.\\
Sezioni dell'ISECOM \emph{Open Source Security Testing Methodology Manual}:
\begin{enumerate}
	\item Information Security
	\item Process Security
	\item Internet Technology Security
	\item Communications Security
	\item Wireless Security
	\item Physical Security
\end{enumerate}
Eseguire security testing in accordo all'OSSTMM richiede che ogni modulo contenuto in ogni sezione sia testato ed elenca sei approcci comuni al testing: dal Double Blind, in cui sia target che attaccante non abbiano informazioni prima di condurre il testing, al Tandem, dove il target e l'attaccante condividono informazioni riguardo il testing apertamente. L'approccio mostrato di seguito è quello Double Blind in quanto è quello che più si avvicina ad una situazione reale.
\subsubsection{Information Security}
In questa sezione risiedono otto moduli:
\begin{enumerate}
	\item Posture assessment
	\item Information integrity review
	\item Intelligence survey
	\item Internet document grinding
	\item Human resources review
	\item Competitive intelligence review
	\item Privacy controls review
	Information controls review
\end{enumerate}
La sezione di Information Security dell'OSSTMM si concentra su \emph{information gathering} e \emph{validation}. Dalla prospettiva di un attaccante, include l'ottenimento e la revisione di informazioni riguardo la marca e modello dello smart meter target per studiarne il funzionamento, gli standard utilizzati e determinare quali attacchi possano essere più adatti rispetto ad altri.


\subsubsection{Process Security Testing}
La seconda sezione dell'ISECOM \emph{Open Source Security Testing Methodology Manual} si concentra sull'analisi di sicurezza dei processi del target e contiene i seguenti cinque moduli:
\begin{enumerate}
	\item Posture review
	\item Request testing
	\item Reverse Request testing
	\item Guided Suggestion testing
	\item Trusted Persons testing
\end{enumerate}

La seconda sezione, quindi, si occupa di ciò che solitamente è chiamata ``social engineering". Ogni modulo punta ad ottenere informazioni da persone attraverso la coercizione e l'inganno. In relazione ad attaccare gli smart meter, ciò include impersonare il tecnico di una compagnia o un consumatore.\\
Con questi metodi, sarebbe possibile ottenere informazioni di valore come specifiche tecniche o amministrative, o credenziali degli utenti.

\subsubsection{Internet Technology Security Testing}
La maggior parte dei moduli applicabili al testing della sicurezza degli smart meter è contenuta nei quattordici moduli di questa sezione:
\begin{enumerate}
	\item Network Surveying
	\item Port Scanning
	\item Services Identification
	\item System Identification
	\item Vulnerability Research and Verification
	\item Internet Application Testing
	\item Router Testing
	\item Trusted Systems Testing
	\item Firewall Testing
	\item Intrusion Detection System Testing
	\item Containment Measures Testing
	\item Password Cracking
	\item Denial of Service Testing
	\item Security Policy Review
\end{enumerate}

\paragraph{Network Surveying}\mbox{}\\
Punta all'identificazione dei sistemi target accessibili in rete. Nel caso di smart meters, questi sono accessibili agli attaccanti attraverso sia reti wireless che le home area networks. In entrambi i casi, il Network Surveying consiste nell'ottenere informazioni sui target (\emph{information gathering}). 
Ciò può essere realizzato in uno dei seguenti modi: passivamente, ascoltando il traffico di passaggio sulla rete, o attivamente, facendo \emph{IP probing} in attesa di una response.\\

\begin{figure}[hbtp]
	\centering
	\includegraphics[scale=.3]{imgs/attack/wireshark.png}
	\caption{Wireshark sniffing tool}
	\label{wireshark_img}
\end{figure}

Per l'identificazione passiva, può essere utilizzato Wireshark figura \ref{wireshark_img}.\cite{wireshark} Wireshark cattura il traffico che attraversa qualsiasi rete in tempo reale e fornisce l'ispezione di centinaia di protocolli. È possibile inoltre specificare gli indirizzi IP di cui effettuare lo sniffing, così da limitare la raccolta di informazione ai possibili target.\\
Per quanto riguarda l'identificazione attiva, invece, è possibile utilizzare la tecnica del \emph{ping sweep}: essa si avvale del Internet Control Message Protocol (ICMP) per individuare i target attivi in rete analizzando le loro response. Nel caso in cui il traffico ICMP sia bloccato, è spesso utilizzato il ping di TCP. Per entrambi i casi è utilizzato il tool di sicurezza Nmap.\cite{nmap}

\paragraph{Port Scanning}\mbox{}\\
Consiste nel fare \emph{probing} sul target in attesa di risposte sulle 65,536 porte TCP e/o UDP. Ottenere una response significa che dei servizi sono in esecuzione sulle porte associate e che potrebbero contenere debolezze che esporrebbero il target. Il tool di port-scanning più utilizzato è Nmap, che permette di effettuare scan TCP completando l'\emph{handshake}, o attraverso scan TCP SYN che utilizza solamente i messaggi iniziali di SYN e SYN ACK dell'\emph{handshake} TCP.\\
L'OSSTMM suggerisce che la scelta di quali delle 65,536 porte da analizzare è a discrezione dell'attaccante e dipende dal contesto.

\paragraph{Services Identification and System Identification}\mbox{}\\
Lo scopo di questi due moduli è di enumerare i servizi in esecuzione sulle porte TCP o UDP che hanno prodotto una response nella fase di port scanning, così come identificare il sistema operativo del target.\\
Entrambi i compiti sono svolti dal tool Nmap, attivando lo switch \textbf{-sV} che fornisce informazioni addizionali ad esempio il numero di versione, come è possibile verificare nella figura \ref{nmap_sv_img}.\\
In questo caso il target utilizzato, \emph{webserver.domain.com}, fornisce informazioni quali il sistema operativo in quanto esposte dal server web Apache. Un attaccante userebbe le informazioni ottenute per verificare la presenza di falle nelle specifiche versioni dei servizi, per poi preparare un attacco.
\begin{figure}[hbtp]
	\centering
	\includegraphics[scale=.3]{imgs/attack/nmap_sv.png}
	\caption{Nmap version detection ouput}
	\label{nmap_sv_img}
\end{figure}

\paragraph{Vulnerability Research and Verification}\mbox{}\\
Una volta noti sistema operativo e servizi in esecuzione con relative versioni si passa alla ricerca  e verifica di vulnerabilità tramite testing manuale ed automatizzato.\\
Un tool comunemente utilizzato per effettuare tale scan è Nessus \cite{nessus}, sviluppato dalla Tenable Network Security mostrato in figura \ref{nessus_img}. Sebbene Nessus sia ottimo per eseguire scanning automatizzato per determinare debolezze come una versione non aggiornata di Apache, il testing manuale dovrebbe coadiuvare quello automatizzato per individuare debolezze che potenzialmente potrebbero essere trascurate.\\
\begin{figure}[hbtp]
	\centering
	\includegraphics[scale=.3]{imgs/attack/nessus.png}
	\caption{Interfaccia web di Nessus}
	\label{nessus_img}
\end{figure}
L'OSSTMM consiglia che il testing automatizzato venga effettuato da almeno due scanner seguito da una verifica manuale. Le tecniche utilizzate per la verifica manuale di vulnerabilità variano molto in funzione della vulnerabilità che si sta cercando, come ad esempio l'uso di Telnet per osservare la versione di uno specifico servizio connettendocisi, o l'utilizzo di un client FTP per connettersi ad un server FTP anonimo.\\
Nel processo di attacco a smart meter, la fase in esame è un passo critico in quanto fornisce i potenziali punti d'ingresso nello smart meter che potrebbero essere sfruttati nel modulo che segue.

\paragraph{Internet Application Testing}\mbox{}\\
Spesso, gli scanner di vulnerabilità non includono la possibilità di eseguire identificazione di vulnerabilità e verifica per Web application.\\
Con l'aumentare delle misure di sicurezza adottate dai produttori di sistemi all'interno del proprio ciclo di sviluppo, il numero di servizi in esecuzione a disposizione di attaccanti si è man mano ridotto. Ciò ha portato questi ultimi a concentrarsi sulle web application in esecuzione sui dispositivi target. Nel caso degli smart meter, tali applicazioni consentono al consumatore di visualizzare o configurare le informazioni di utilizzo, o permettono ai tecnici di configurare il dispositivo. Lo scopo di questo modulo è lo stesso del precedente, ma in un ambiente differente.\\
Effettuare identificazione e verifica di vulnerabilità su web application è significativamente più complesso che eseguire lo stesso test su servizi in esecuzione. Questo è il risultato del livello di personalizzazione trovato in ogni web application: è raro il caso che una web application sia esattamente come un'altra, ed anche nel caso di due \emph{webapp} identiche trovate in esecuzione, la loro infrastruttura di backend potrebbe differire. Per questo il testing manuale gioca un ruolo significativo e i tool a supporto di tale operazione sono molteplici. L' Open Web Application Security Project (OWASP) ha sviluppato una guida al testing per le web application, disponibile a questo indirizzo \url{https://www.owasp.org/index.php/Category:OWASP_Testing_Project}.

\paragraph{Password Cracking}\mbox{}\\
Questo modulo consiste nell'individuazione delle credenziali valide di un servizio in esecuzione o di una web application. Il testing può essere effettuato sia a partire da una lista precompilata di password, noto come attacco basato su dizionario, che provando ogni possibile combinazione di un certo alfabeto di caratteri, noto come attacco brute force.\\
Quando si esegue password cracking, è bene tenere a mente che molti servizi e web application implementano un servizio di blocco temporaneo o permanente, che disabilita un account se si verificano troppi tentativi di accesso con password invalide durante un determinato periodo di tempo.\\
Un tool comunemente utilizzato nell'ambito del password cracking, che supporti sia attacchi con dizionario che brute force, è Cain \& Abel \cite{cainabel}, mostrato in figura \ref{cainabel_img}.\\
\begin{figure}[hbtp]
	\centering
	\includegraphics[scale=.3]{imgs/attack/cainabel.png}
	\caption{Il tool di password cracking Cain \& Abel}
	\label{cainable_img}
\end{figure}
Il password cracking ha un ruolo fondamentale nell'attacco ad uno smart meter quando ci si trova davanti ad un prompt di autenticazione. Se un attaccante riesce ad ottenere le credenziali di uno smart meter, egli può istantaneamente avere accesso al dispositivo.

\paragraph{Denial of Service Testing}\mbox{}\\
Il modulo di Denial of Service si occupa di identificare i punti deboli che potrebbero essere del device stesso o all'interno dell'infrastruttura sottostante. Tale operazione potrebbe coinvolgere l'utilizzo degli strumenti precedentemente descritti, Wireshark e Nmap. Ad esempio, Wireshark potrebbe essere utilizzato per determinare i regolari pattern di traffico da e verso lo smart meter. Nmap invece potrebbe essere utilizzato per incrementare gradualmente il traffico verso lo smart meter nell'intento di sovraccaricare il dispositivo o la sua infrastruttura.\\
Se l'obiettivo dell'attaccante è semplicemente di negare il servizio ad uno smart meter, sarebbe molto semplice condurre un attacco del genere se comparato con attacchi che mirano alla compromissione della confidenzialità e/o integrità dello smart meter.

\paragraph{Exploit Testing}\mbox{}\\
Tutti i moduli descritti fin'ora gettano le basi per l'esecuzione dell'exploit testing. L'exploit testing punta ad utilizzare le vulnerabilità identificate per compromettere lo smart meter. Esempio di exploit testing includono l'utilizzo di codice per sfruttare un buffer overflow in un servizio in esecuzione o utilizzando SQL injection per accedere ad una shell di comando attraverso una falla nella validazione di un input all'interno di una web application.\\
Metasploit è un exploit tool disponibile gratuitamente che offre ai tester di sicurezza e agli attaccanti un considerevole numero di \emph{vulnerability exploit} e \emph{payload}. \cite{metasploit}
\begin{figure}[hbtp]
	\centering
	\includegraphics[scale=.3]{imgs/attack/metasploit.png}
	\caption{Il tool di vulnerability exploit Metasploit}
	\label{metasploit_img}
\end{figure}
Per un security tester, lo scopo finale è spesso quello di compromettere il target, laddove per un attaccante, la sola compromissione dello smart meter possa essere un altro passo nella propria metodologia personale di raggiungere l'obiettivo preposto.



\section{False Data Injection}
Nelle moderne reti di Smart Grid, la rete elettrica tradizionale è potenziata dai più recenti progressi nei campi del sensing, della misurazione e dei dispositivi di controllo con una comunicazione bidirezionale tra produttore e consumatore. La produzione di elettricità, la sua trasmissione, la distribuzione ed il consumo, scambiano informazioni sullo stato della griglia che vengono recapitate agli utenti del sistema, agli operatori ed ai dispositivi.\\
La stima dello stato ha una funzione chiave nella costruzione di modelli \emph{real-time} della rete elettrica nei centri di gestione dell'energia (EMC) \cite{monticelli}. Un modello \emph{real-time} è una rappresentazione matematica quasi statica delle attuali condizioni all'interno di una rete elettrica interconnessa. Questa rappresentazione matematica è solitamente ottenuta dai dati provenienti dalle misurazioni e dalla telemetria che avvengono a distanza di pochi secondi nel centro di controllo dell'energia (ECC). Modelli \emph{real-time} della rete possono essere utilizzati per fare scelte ottimali, rispettando vincoli tecnici come congestione delle linee di trasmissione, voltaggio e stabilità transiente. In pratica, sia economicamente che in termini di fattibilità, non è possibile misurare tutti i possibili stati nella rete; quindi, la stima dello stato è uno strumento utile per stimare tali quantità, a partire da un insieme limitato di misurazioni. Solitamente sono usati due tipi di informazione per la stima dello stato:
\begin{itemize}
	\item Dati analogici di sistemi come flussi Megavar sulle linee principali, il carico \emph{P} e \emph{Q} sui generatori e trasformatori, e i voltaggi dei bus di sistema
	\item Lo stato on/off dei dispositivi di switch come interruttori di circuito, switch di disconnessione, e tap dei trasformatori che determinano la topologia di rete
\end{itemize}
A causa dell'importanza della stima dello stato, gli effetti negativi dell'\emph{iniettare} misurazioni scorrette sono studiati in letteratura \cite{baddatainj}.  Le misurazioni scorrette possono avvenire a causa di anormalità di misura impreviste, o iniezioni dovute ad attacchi malevoli. Ad esempio, \cite{falsedatainj} è il lavoro pionieristico nello studio degli attacchi di \emph{bad data injection} che non possono essere rilevati (chiamati \emph{stealth attacks}), e mostra come un attaccante possa portare a compimento tali attacchi ``\emph{stealth}" falsificando le misure del flusso elettrico alle unità terminali remote (RTUs), manomettendo l'eterogenea rete di comunicazione o infiltrandosi all'interno del sistema di controllo della supervisione di controllo e dell'acquisizione dati (SCADA) attraverso la LAN dell'ufficio del centro di controllo. Si consideri che un sistema SCADA o un sistema mi misurazione su larga area (WAMS) ottiene informazioni sulla rete elettrica (valori delle misure, stato degli interruttori, ecc.) a specifici tempi e luoghi. I centri di controllo usano le informazioni collezionate per scopi diversi, ad esempio la risoluzione di un problema di stima dello stato. In \cite{baddatainj2}, viene mostrata la fattibilità degli  attacchi di bad data injection non rilevabili, con l'obiettivo di manipolare i prezzi del mercato elettrico.
\begin{figure}[h]
	\centering
	\includegraphics[scale=.3]{imgs/attack/fourbuspowernet.png}
	\caption{Illustrazione di una rete elettrica a quattro bus, centro di controllo, varie funzioni principali (AGC, OPF, EMS), e l'operatore. G rappresenta un generatore, il punto nero rappresenta misurazioni attive sul flusso di corrente ed il triangolo sul bus rappresenta il carico della regione o città.}
	\label{fourbuspowernet_img}
\end{figure}

\subsection{Stima dello stato e Bad Data Injection}
I sistemi di corrente in generale consistono di tre sottosistemi: generazione, trasmissione e distribuzione. Le linee di trasmissione sono utilizzate per trasmettere la corrente elettrica generata ai consumatori. In teoria, la corrente complessiva trasmessa tra il bus \emph{i} ed il bus \emph{j} dipende dalla differenza di voltaggio tra i due bus, ed è funzione dell'impedenza tra questi bus. In genere, le linee di trasmissione hanno un alto rapporto reattanza/resistenza ($X/R$), e quindi l'impedenza di una trasmissione può essere approssimata con la sua reattanza. La corrente attiva trasmessa dal bus \emph{i} al bus \emph{j} può essere scritta come 
\\
\\
\indent$P_{ij} = \frac{V_{i}V_{j}}{X_{ij}}sin(\theta_{i} - \theta_{j})$,
\\
\\
dove $V_{i}$ è la tensione, $\theta_{i}$ è l'angolo di fase del voltaggio nel bus \emph{i}, ed $X_{ij}$ è la reattanza della linea trasmissiva tra il bus \emph{i} ed il bus \emph{j}.\\
Negli studi del flusso di corrente DC (che in questo caso sta per linearità delle equazioni piuttosto che corrente diretta), solitamente si assume che le differenze di fase tra due bus siano piccole, e che le ampiezze dei voltaggi nei bus siano vicine all'unità (dopo essere state normalizzate). Per cui, un'ulteriore semplificazione porta ad una relazione lineare tra gli angoli di fase e la reattanza delle linee,
\\
\\
\indent$P_{ij} = \frac{\theta_{i}\theta_{j}}{X_{ij}}$.
\\
\\
Negli studi sui flussi di potenza, l'angolo di fase del voltaggio ($q_i$) del bus di riferimento è fissato e noto; quindi, solamente $n - 1$ angoli devono essere stimati. I vettori di stato sono definiti come $\textbf{x} = [\theta_1, \ldots, \theta_n]^T$, cioè il vettore degli $n$ angoli di fase dei bus $\theta_i, i = 1, \ldots, n$.\\
Il problema della stima dello stato consiste nello stimare gli $n$ angoli di fase $\theta_i$, osservando $m$ misure in tempo reale, denotate dal vettore $\textbf{z}$ al centro di controllo. Queste misure potrebbero essere sia di corrente attiva trasmessa dal bus $i$ al $j$, $P_{ij}$, sia di corrente attiva al bus $i$, $P_i$. La corrente attiva iniettata nel bus $i$ è la super composizione della corrente trasmessa tramite le linee connesse al bus $i$ come $P_i = \sum_j{P_{ij}}$. Il vettore delle osservazioni $\textbf{z}$ puuò essere descritto come $\textbf{z} = \textbf{h(x)} + \textbf{e}$, dove $\textbf{h(x)}$ è la relazione non lineare tra le misure $\textbf{z}$ e lo stato del sistema $\textbf{x}$, ed $\textbf{e} = [e_1, \ldots, e_m]^T$ è il vettore del rumore Gaussiano delle misure con matrice di covarianza $\Sigma_e$.\\
La matrice del Jacobiano $\textbf{H} \in {\rm I\!R}$ è definita come\\
\indent$\textbf{H}=\frac{\partial\textbf{h(x)}}{\partial\textbf{x}}|_{x=0}$.\\
Se la differenza di fase è piccola, il modello di approssimazione lineare della misura di corrente può essere descritto come\\
\indent Misura sotto Operazioni Normali: $\textbf{z} =  \textbf{Hx} + \textbf{e}$.\\
Da notare che $\textbf{H}$ è generalmente sconosciuta agli attaccanti ma nota all'ISO. Date le misure sul flusso di corrente, il vettore di stato stimato $\widehat{x}$ può essere computato come $\widehat{\textbf{x}} = (\textbf{H}^T\sum_e^{-1}\textbf{H})-1\textbf{H}^T\sum_e^{-1}\textbf{z}$.\\
La figura \ref{fourbuspowernet_img} mostra il sistema di test a quattro bus della IEEE: ogni bus ha il corrispondente voltaggio ($V_q$) e l'angolo di fase ($\theta_q$); il centro di controllo invia i dati delle misurazioni ($z_{qr}$) ed in seguito lo stimatore di stato inferisce gli stati del sistema che possono essere utilizzati in differenti funzioni, come ad esempio il controllo della generazione automatico (AGC), il flusso di controllo ottimale (OPF), ed il sistema di gestione dell'energia (EMS). L'operatore effettua la decifsione finale per il controllo dei generatori e la gestione del carico (per bilanciare la fornitura e la domanda).
\subsection{Bad Data Detection}
Nella stima dello stato di un sistema elettrico, i ``bad data" come ad esempio bias di misurazione, derive di misura o connessioni errate devono essere identificate. Con il \emph{bad data injection}, gli attaccanti possono iniettare dati all'interno del vettore di misure $r$ ed il sistema può essere descritto come\\
\indent Misura sotto Attacco non-Stealth: $\textbf{z}^\prime = \textbf{H}(\textbf{x}) + \textbf{b} + e$, $\textbf{a} = I\textbf{b}$.\\
Si definisce il vettore residuo $\textbf{r}$ come la differenza tra le qualità misurate ed i valori calcolati dagli stati stimati, precisamente, $\textbf{r} = \textbf{z} - \textbf{H}\widehat{\textbf{x}}$. La media e la covatianza del residuale sono rispettivamente $E(\textbf{r}) = 0$, e $cov(\textbf{r}) = I\sum_{\epsilon}$, dove $I = \textbf{I} - \textbf{M}$, ed $\textbf{M} = \textbf{H}(\textbf{H}^T\sum_e^{-1}\textbf{H})-1\textbf{H}^T\sum_e^{-1}$.\\
I minimi quadrati pesati dell'errore di misura $r^{T}\sum_{e}^{-1}r$ segue la distribuzione del chi-quadro con $n - m$ gradi di libertà \cite{monticelli}. L'ipotesi a riguardo del rilevamento dei dati fasulli può essere espressa come, $\textbf{r}^T\sum_{e}^{-1}\textbf{r} \ggll \chi_{n-m}^{2},\zeta$, dove $\zeta$ è la probabilità della confidenza del rilevamento.\\
I residuali normalizzati, di tutte le misure, sono utilizzati per identificare i ``bad data". Se la misura corrispondente al maggior residuale normalizzato è maggiore di una soglia di identificazione fissata $\gamma$, \\
\indent $max_i(|\textbf{r}_i|/\sqrt{cov(\textbf{r})}) \geq \gamma$,\\
allora quella misura è considerata come un dato fasullo ed è eliminato dalla stima dello stato.
\subsection{Stealth Bad Data Injection}
Utilizzando lo schema di rilevamento discusso precedentemente, il centro di controllo può difendersi da attacchi \emph{naive} di data injection ed identificare la sorgente di dati corrotti. Questo tipo di attacchi sono chiamati \emph{non stealth}. Però, se un attaccante ha conoscenza della topologia \textbf{H}, può iniettare dati fasulli della forma $\textbf{H}\delta\textbf{x}$ nella misura $\textbf{r}$, più precisamente,\\
\indent Misura sotto Attacco Stealth: $\textbf{z}^\prime = \textbf{H}(\textbf{x} + \delta\textbf{x}) + e$.\\
In questo caso, il test dell'ipotesi fallirebbe il rilevamento dell'attaccante, ed il centro di controllo crederebbe che il vero stato sia $\textbf{x} + \delta\textbf{x}$. Questo è chiamata \emph{stealth bad data injection}. Un'assunzione critica per la fattibilità di questo tipo di attacchi è la disponibilità di informazione completa sulla topologia. Tale assunzione può essere rilassata dal punto di vista dell'attaccante.
\subsection{Meccanismo Difensivo}
La strategia difensiva presentata in \cite{baddatainjattackdef} si basa sull'analisi statistica online della sequenza di dati simultanea al controllo del ritardo di rilevamento e la probabilità di errore entro i livelli desiderati. I metodi di stima dello stato convenzionali \cite{convdef1}, \cite{convdef2} per la \emph{bad data detection} utilizzano le misure per bilanciare il tasso di falsi allarme o il rapporto dei rilevamenti persi. Invece, l'approccio presentato in \cite{baddatainjattackdef} punta a minimizzare il delay di rilevamento soggetto al vincolo sulla probabilità di errore.\\
Si rappresenti con $z_t$ il vettore di osservazioni $m$-dimensionale al tempo $t$. In assenza di un avversario, $z_t$ può essere modellato ,per trattabilità, come una distribuzione Gaussiana multivariata a media zero $\mathcal{N}(0, \sum_z)$. Si assume che l'avversario sia inattivo inizialmente; ad un tempo casuale sconosciuto $t$, diventa attivo ed inietta dati malevoli. L'ipotesi binaria può essere formulata come $\mathcal{H}_0: \textbf{Z}_t \sim \mathcal{N}(0, \sum_t)$ ed $\mathcal{H}_1: \textbf{Z}_t \sim \mathcal{N}(\textbf{a}_t, \sum_z)$, dove $\textbf{a}_t = [a_{t,1}, a_{t,2}, \ldots, a_{t,m}]^T \in R^m$ è il vettore dei dati malevoli sconosciuti iniettati dall'attaccante al tempo $t$, ed $\sum_z$ è $\textbf{H}\sum_x\textbf{H}^T + \sum_e$. In altre parole, si vuole rilevare un cambiamento nella distribuzione da $\mathcal(N)(0, \sum_z)$ a $\mathcal(N)(\textbf{a}_t, \sum_z)$ ad un tempo non noto $t$ con $\textbf{a}_t$ sconosciuto.\\
Sia $T_h$ lo \emph{stopping time}, il tempo in cui viene rilevato il cambiamento. Se $T_h < \tau$, è un falso allarme. La lunghezza di esecuzione media (ARL) è $T_d = E[T_h - \tau]$. Basandosi sulla formulazione di Lorden \cite{lorden}, è possibile minimizzare il delay nel caso peggiore, che può essere descritto come $T_d = sup_{\tau \geq 1} E_{\tau}[T_h - \tau | T_d \geq \tau]$. Per computare il minimo $T_d$, l'algoritmo CUSUM di Page è la miglior tecnica per affrontare questo tipo di problemi \cite{lorden}. La maggior parte dei modelli basati su CUSUM assume la conoscenza perfetta delle funzioni di likelihood. Nella detection della bad data injection, i parametri della distribuzione $\mathcal{H}_1$ non possono essere completamente definiti a causa dei parametri degli attaccanti e del modello statistico che non sono noti. Per cui, bisogna progettare meccanismi per il più rapido rilevamento in presenza di parametri non noti.\\
Il CUSUM test adattivo è ricorsivo per sua natura. Ogni ricorsione comprende due step interfogliati:
\begin{itemize}
	\item test CUSUM multi-thread
	\item risolutore lineare di parametri non noti
\end{itemize}
Il CUSUM test multi-thread estende l'algoritmo di Page. Esso considera il tasso di likelihood di $m$ misure a tempo $t$ così da determinare il tempo di stop $T_h$, che può essere descritto come $T_h = Inf\lbrace t \geq 1 | S_t > h \rbrace$, in cui la soglia di rilevamento $h$ è una funzione del tasso di falso allarme (FAR), del tasso di rilevamenti persi (MDR), e la varianza del processo, con statistiche cumulative al tempo $t$: $S_t = max_{1\geq k \geq T_h}\sum_{t=k}^{T_h}$, con $L_t$ pari alla somma della funzione del tasso di likelihood per tutte le misure $(z_{t,j}, j \in \lbrace1, 2, \ldots, m\rbrace)$ al tempo $t$. È possibile esprimere $L_t(\textbf{Z}_t)$ come\\
\indent $\sum_{j=1}^{m}log\frac{f_1(z_{t,j})}{f_0(z_{t,j})}$,\\
dove $f_1(z_{t,j})$ e $f_0(z_{t,j})$ corrispondono alla distribuzione della $j$-esima osservazione al tempo $t$ sotto attacco. Al tempo $t$, la statistica cumulativa $S_t$ può essere risolta ricorsivamente come $max[0, S_{t-1} + L_t(\textbf{Z}_t)]$, dove $S_0 = 0$ quando $t = 0$. Il centro di controllo fa scattare un allarme quando l'accumulazione supera una determinata soglia $h$.\\
\begin{figure}[h]
	\centering
	\includegraphics[scale=.3]{imgs/attack/adaptive_CUSUM.png}
	\caption{Il test CUSUM adattivo con intervallo di decisione.}
	\label{cusum_img}
\end{figure}
A causa di un modello statistico avversario sconosciuto, il test del tasso di likelihood generalizzato (GLRT) può essere utilizzato nell'algoritmo CUSUM di Page \cite{lorden}. L'idea è quella di applicare GLRT sostituendo il parametro sconosciuto secondo la stima a maximum likelihood. Quindi, l'espressione ricorsiva del CUSUM test non è più valida in quanto GLRT ha bisogno di computare ogni elemento non noto di ogni misurazione al tempo $t$ stimandolo dalle osservazioni fino al tempo corrente $t$. In altre parole, GLRT richiede la memorizazione delle osservazioni e l'esecuzione della stima ML dei parametri sconosciuti ad ogni punto temporale. Per cui, GLRT è troppo costoso computazionalmente da implementare in pratica per effettuare rilevamento veloce.\\
Per ridurre la complessità computazionale, è possibile applicare il test Rao \cite{lorden}, che è un modello di test asintoticamente equivalente a GLRT. Il Rao test computa le derivate rispetto al parametro non noto valutato in zero, e può essere implementato efficientemente. Inoltre, il Rao test non coinvolge la complessa computazione della stima a maximum likelihood.\\
\subsection{Strategia di attacco}
Gli attacchi stealth sono attuabili quando gli attaccanti hanno piena conoscenza della topologia. Una domanda importante che sorge spontanea è: \textit{se la topologia non è disponibile, può un attaccante effettuare comunque stealth bad data injection?} Sorprendentemente, \textit{si}. L'idea principale presentata in \cite{esmalifalak}, si basa su parametri del sistema variabili in un piccolo range dinamico, infatti, l'informazione sulla topologia è incorporata nelle correlazioni tra le misure di flusso di potenza. Siano $\textbf{z}(t)$ e $\textbf{x}(t)$ le misure ed i vettori di stato al tempo $t$, dove $\textbf{x}(t)$ è sconosciuto. Ad un certo tempo $t$, è impossibile inferire $\textbf{H}$ solamente a partire da $\textbf{z}(t)$. Comunque, con il passare del tempo, e la conoscenza delle proprietà stocastiche del processo casuale $\textbf{x}(t)$, si potrebbe essere in grado di inferire \textbf{H}.\\
Nei sistemi elettrici, le variabili di stato sono generalmente una funzione non lineare dei carichi \textbf{y} e della topologia \textbf{H: x = } \emph{f}(\textbf{y, H}). Mentre la topologia è nota per essere statica in un determinato periodo temporale, i carichi possono essere modellati come \textit{indipendentemente} variabili. Se tali variazioni sono sufficientemente piccole, \emph{f} può essere approssimata utilizzando $\textbf{x} = \textbf{Ay}$, dove \textbf{A} è la matrice dei coefficienti di primo ordine dell'espansione di Taylor in \textbf{y} ($\textbf{z = HAy + e}$).\\
Con \textbf{HA} ed \textbf{y}, è possibile portare a compimento l'attacco modificando i dati misurati come $\textbf{z}^\prime + \textbf{HA}\delta\textbf{y}$, dove $\delta\textbf{y}$ è scelto arbitrariamente. 
Il vettore di stato è stimato come $\widehat{\textbf{x}} = (\textbf{H}^T\sum_e^{-1}\textbf{H})-1\textbf{H}^T\sum_e^{-1}\textbf{z}^\prime$. Sia $\delta\textbf{x} = A\delta\textbf{y}$. Siccome $r = \textbf{z}^\prime - \textbf{H}\wedge x = \textbf{z} + \textbf{H}(\widehat{\textbf{x}} + \delta\textbf{x}), E(\textbf{r}) = 0, cov(\textbf{r}) = (\textbf{I} - \textbf{M})\sum_e$. In altre parole, la media e la varianza di \textbf{r} sono le stesse del caso senza attaccanti. Quindi, utilizzando li metodo delresiduo massimo, l'attacco non può essere rilevato.\\
Per inferire \textbf{HA} ed \textbf{y}, è possibile adottare la tecnica della \emph{linear independent component analysis} (ICA).
\begin{figure}[htbp]
	\centering
	\fbox{\begin{minipage}{15em}
		Input: \textbf{z} = data matrix
		\begin{enumerate}
			\item $\lbrack \textbf{G}$ and $\textbf{y} \rbrack$ = FastICA(\textbf{z})
			\item If max(\textbf{z} - \textbf{Gy})  $>$ $\epsilon$ then exit
			\item Generate $\delta$\textbf{y} $\sim N(0, \sigma^2)$
			\item $\textbf{z}^\prime$ = \textbf{z} + \textbf{G}(\textbf{y} + $\delta$\textbf{y})

		\end{enumerate}
		Output: false data $\textbf{z}^\prime$
		\end{minipage}
	}
	\caption{Stealth false data injection.}
	\label{algo1}
\end{figure}	


\section{Time Delay Attack}
\section{Replay Attack}
\section{Jamming}
% \section{Web Application} già discussa nell'OSSTMM
\section{Rilevamento}
\section{Power Fingerprinting}
%\section{Optimal Malicious Attack Construction and Robust Detection in Smart Grid Cyber Security Analysis}


\begin{thebibliography}{99}
\bibitem{standard14908} International Organization for Standardization. ISO/IEC 14908-1:2012:Information technology – Control network protocol – Part 1: Protocol stack, 2012.
\bibitem{wep1} Andreas Klein. \emph{Attacks on the RC4 stream cipher.} Des. Codes Cryptography, 48(3):269–286, 2008.
\bibitem{wep2} Erik Tews, Ralf-Philipp Weinmann, and Andrei Pyshkin. \emph{Breaking 104 bit WEP in less than 60 seconds.} In Sehun Kim, Moti Yung, and Hyung-Woo Lee, editors, Information Security Applications, 8th International Workshop, WISA 2007, Jeju Island, Korea, August 27-29, 2007, Revised Selected Papers, volume 4867 of Lecture Notes in Computer Science, pages 188–202. Springer, 2007.
\bibitem{aircrackng} Aircrack-ng. \url{http://www.aircrack-ng.org/}
\bibitem{kali} Kali Linux. \url{https://www.kali.org/}
\bibitem{wireshark} Wireshark. \url{https://www.wireshark.org/}
\bibitem{nmap} Nmap. \url{https://www.nmap.org/}
\bibitem{nessus} Nessus. \url{https://www.nessus.org/}
\bibitem{cainabel} Cain \& Abel. \url{http://www.oxid.it/}
\bibitem{metasploit} Metasploit. \url{http://www.metasploit.com/}
\bibitem{monticelli} A. Monticelli, ``Electric Power System State Estimation," Proc. IEEE, vol. 88, Feb. 2000, pp. 262–82.
\bibitem{baddatainj} M. Esmalifalak, Z. Han, and L. Song, ``Effect of Stealthy Bad Data Injection On Network Congestion In Market Based Power System," IEEE WCNC 2012, Paris, France, Apr. 2010.
\bibitem{falsedatainj} Y. Liu, M. K. Reiter, and P. Ning, ``False Data Injection Attacks Against State Estimation in Electric Power Grids," 16th ACM Conf. Computer and Commun. Security, Gaithersburg, MD, Nov. 2009, pp. 21–30.
\bibitem{baddatainj2} L. Xie, Y. Mo, and B. Sinopoli, ``False Data Injection Attacks in Electricity Markets," 1st IEEE Int’l. Conf. Smart Grid Commun., Gaithersburg, MD, Oct. 2010, pp. 226–31.
\bibitem{baddatainjattackdef} Huang, Yi, et al. ``Bad data injection in smart grid: attack and defense mechanisms." Communications Magazine, IEEE 51.1 (2013): 27-33.
\bibitem{convdef1} A. Abur and A. G. Exposito, Power System State Estimation: Theory and Implementation, Marcel Dekker, 2004.
\bibitem{convdef2} A. J. Wood and B. F. Wollenberg, Power Generation, Operation, and Control, Wiley, 1996.
\bibitem{lorden} H. V. Poor and Q. Hadjiliadis, Quickest Detection, Cambridge Univ. Press, 2008.
\bibitem{esmalifalak} M. Esmalifalak et al., ``Stealth False Data Injection using Independent Component Analysis in Smart Grid," 2nd IEEE Conf. Smart Grid Commun., Brussels, Belgium, Oct. 2011.
\end{thebibliography}
%----------------------------------------------------------------------------------------
%	REFERENCE LIST
%----------------------------------------------------------------------------------------

\begin{thebibliography}{99}
\bibitem{securingSG} Tony Flick and Justin Morehouse. \emph{Securing the Smart Grid. Next Generation Power Grid Security.} Elsevier Inc, 2011.
\bibitem{smartgrids} Stuart Borlase. \emph{Smart Grids: Infrastructure, Technology, and Solutions.} CRC Press, 2013.
\bibitem{smartgrid} Janaka Ekanayake, Kithsiri Liyanage, Jianzhong Wu, Akihiko Yokoyama and Nick Jenkins. \emph{Smart Grid: Technology and Applications.} Wiley, 2012.
\bibitem{pathsg} Hassan Farhangi. The Path of the Smart Grid. In \emph{IEEE power \& energy magazine}, pages 18-28, january/february 2010.
\bibitem{surveymotiv} Ye Yan, Yi Qian, Hamid Sharif, and David Tipper. A Survey on Smart Grid Communication Infrastructures: Motivations, Requirements and Challenges. In \emph{IEEE communications surveys \& tutorials,} vol.15, no.1, 2013.
\bibitem{surveyrouting} Nico Saputro, Kemal Akkaya and Suleyman Uludag. A survey of routing protocols for smart grid communications. In \emph{Computer Networks 56}, pages 2742–2771, 2012.
\bibitem{improvedgrid} Xi Fang, Satyajayant Misra, Guoliang Xue and Dejun Yang. Smart Grid – The New and Improved Power Grid: A Survey. In \emph{IEEE communications surveys \& tutorials,} vol.14, no.4, 2012.
\end{thebibliography}

%----------------------------------------------------------------------------------------

\end{document}
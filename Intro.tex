L'infrastruttura elettrica attuale, non subisce modifiche da circa un centinaio d'anni: le componenti della rete, organizzate in una struttura gerarchica, sono vicine alla fine della loro vita. \newline Mentre la rete invecchia sempre più, la richiesta di energia elettrica aumenta gradualmente e l'attuale organizzazione è troppo complessa e poco adatta per soddisfare i bisogni del 21-esimo secolo. \newline Tra le mancanze dell'infrastruttura corrente vi sono: mancanza di analisi automatizzate, tempi di risposta lenti, mancanza di consapevolezza della situazione, ecc. A tali fattori, si uniscono anche l'aumento della popolazione sul pianeta e la conseguente richiesta di energia, il cambiamento del clima globale, i fallimenti delle apparecchiature che costituiscono la rete, i problemi di conservazione dell'energia, capacità limitata della generazione di energia, comunicazione unidirezionale, diminuzione di combustibili fossili e scarse capacità di recupero in caso di guasti. \newline È facile capire, analizzando tutti questi fattori, che vi è un bisogno urgente di una nuova infrastruttura elettrica in grado di risolvere tutti questi problemi. \newline \newline
L'attuale rivoluzione dei sistemi di comunicazione, particolarmente stimolata anche dalla crescita di Internet, offre la possibilità di migliorare il monitoraggio e, in generale, le funzionalità dei sistemi energetici e di rendere, quindi, le operazioni più efficaci e flessibili ma, allo stesso tempo, meno costose. \newline
La \textbf{Smart Grid} è l'opportunità per utilizzare le novità introdotte dall'ICT   (\textit{Information and Communication Technology}) al fine di rivoluzionare il sistema energetico. Tuttavia, a causa delle grandi dimensioni sia del sistema che dell'entità degli investimenti che sono stati fatti nel corso degli anni, qualsiasi cambiamento significativo sarà costoso e richiederà un'attenta giustificazione. \newline La Smart Grid è una rete elettrica moderna che offre migliore efficienza, affidabilità e sicurezza, permettendo anche una facile integrazione di nuove fonti di energia rinnovabile. \newline \newline
In confronto ai sistemi precedenti, la Smart Grid è concepita per integrare pienamente, all'interno di milioni di dispositivi, tecnologie che permettano comunicazioni veloci e bidirezionali e che permettano di formare un'infrastruttura dinamica ed interattiva con nuove e migliori capacità di gestione dell'energia. Tuttavia, una dipendenza così forte dal \textit{networking} di informazioni, inevitabilmente sottopone la Smart Grid ad una serie di potenziali vulnerabilità associate ai sistemi di comunicazione e di rete. Ciò, in pratica, aumenta il rischio di compromettere l'affidabilità e la sicurezza delle operazioni dell'infrastruttura che costituiscono gli obiettivi principali della Smart Grid. Per esempio, una potenziale intrusione nella rete da parte di un individuo non autorizzato, potrebbe portare ad una serie di conseguenze negative che vanno dall'\textit{information leakage} alla generazione di fallimenti in cascata, come ad esempio un blackout totale e la distruzione dell'intero sistema. \newline Pertanto, lo scopo di questa tesina è analizzare i problemi legati alla \textbf{sicurezza} della Smart Grid, che è critica nella progettazione delle reti di comunicazione ed è considerata una delle più alte priorità nello sviluppo della rete elettrica moderna.
\newline \newline
La tesina è strutturata nel seguente modo:
\begin{itemize}
\item Capitolo 2
\item Capitolo 3
\item Capitolo 4
\item Capitolo 5
\item Capitolo 6
\end{itemize}
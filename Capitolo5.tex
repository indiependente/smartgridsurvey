Negli ultimi anni, un numero sempre crescente di infrastrutture critiche è stato digitalizzato, aggiungendo capacità di comunicazione e di computazione a numerosi dispositivi nelle reti di distribuzione dell'energia e dell'acqua, sistemi di trasporto, e manifatturieri. Ciò avviene in parte con lo scopo di aumentare l'efficienza, ma spesso è anche un requisito necessario a gestire l'ambiente che muta, come ad esempio la generazione locale dell'energia o il passaggio ai veicoli elettrici nel caso della distribuzione energetica.\\
Un progetto di digitalizzazione molto visibile è il passaggio dai \emph{meter} analogici ai digitali (\emph{smart}), che è attualmente in corso in vari paesi del mondo. Oltre ad una fatturazione complessivamente più precisa, uno smart meter può anche dare input agli algoritmi di controllo della grid, essere usato nei mercati energetici, comunicare con la \emph{smart home} (ad esempio, per regolare l'aria condizionata ed i sistemi di riscaldamento quando la richiesta energetica è alta), oppure per disconnettere da remoto un consumatore. In questo modo, un dispositivo precedentemente disconnesso e non critico si trasforma in un dispositivo connesso che può generare dati \emph{process-critical}.\\
La robustezza dei dati e dei comandi di switch è vitale - se una grande quantità di famiglie viene disconnessa simultaneamente, l'energia in eccesso non ha dove andare, potrebbe danneggiare la grid. In maniera simile, se gli algoritmi di manutenzione si basano su dati provenienti da misure effettuate dagli smart meter, input errati possono produrre effetti di gran lunga peggiori delle frodi di fatturazione. Ciò pone una nuova sfida per i produttori di meter: progettare dispositivi economici, largamente distribuiti e che lavorino su canali con banda banda molto ristretta.
\section{Open Smart Grid Protocol}
L'Open Smart Grid Protocol (OSGP) è stato uno dei primi protocolli di comunicazione su powerline per smart meters disponibile sul mercato, ed è largamente utilizzato per comunicare tra smart meter e l'aggregatore di dati, il quale è un dispositivo che colleziona dati provenienti da diverse centinaia di meter in un segmento di PLC (Power Line Communication) e li inoltra ad un controllore centralizzato.\\
L'OSGP risiede nel layer delle applicazioni dello stack protocollare definito dallo standard 14908 ISO/IEC \cite{standard14908}. Nonostante l'OSGP sia principalmente utilizzato per applicazioni di smart-metering, esso è stato progettato per un utilizzato più ampio all'interno di dispositivi della Smart Grid. Lo stack protocollare è molto leggero. Tale leggerezza è ottenuta pagando in sicurezza, infatti le primitive crittografiche consigliate dal NIST  (ad esempio: Advanced Encryption Standard - AES, in \emph{authenticated mode}) sono evitate, optando per altre meno intense computazionalmente: lo stream cipher RC4 per la cifratura ed una funzione digest non standard per l'autenticazione dei messaggi.\\
Il problema più grande è la combinazione di uno stream cipher con una funzione di digest lineare, che apre le possibilità per attacchi alle chiavi crittografiche ed ai messaggi. Ad esempio, si potrebbe sfruttare tale vulnerabilità utilizzando il ricevente di un messaggio come un ``oracolo". Dato un messaggio propriamente cifrato ed autenticato, un attaccante ipotizza una combinazione di un bit del messaggio ed un bit della chiave, invia un messaggio al ricevente appositamente confezionato, ed utilizza la risposta del  dispositivo che rifiuti o confermi l'ipotesi. Siccome questo può essere effettuato per singoli bit, un attaccante può ricostruire la chiave principale del dispositivo (OMA key) con al più $96 \times 3$ risposte.
\subsection{Potenziali debolezze del protocollo}
Sono state identificate quattro potenziali debolezze nel protocollo OSG.
\subsubsection{Utilizzo di RC4}
Negli ultimi anni sono state identificate varie debolezze nello stream cipher RC4. Uno dei problemi è la correlazione tra la chiave ed il \emph{keystream} di RC4 che può essere utilizzata per ricostruire la chiave. Questa debolezza è stata sfruttata con successo attraverso i vettori di inizializzazione (IVs)  utilizzati per ``rompere" WEP (Wired Equivalent Protection), standard che fa largo uso di RC4. Gli attacchi \cite{wep1}, \cite{wep2} rompono WEP in pochi secondi e tool per l'attacco come Aircrack-ng \cite{aircrackng} sono disponibili gratuitamente per effettuare penetration testing \cite{kali}.\\
L'uso di RC4 nel protocollo OSG è simile al modo in cui RC4 era usato nello standard WEP. Mentre, per ogni messaggio, è generata una nuova chiave, solamente i primi 8 byte della chiave variano, ed i restanti 8 byte sono costanti. Inoltre, gli 8 byte iniziali vengono messi in XOR con un valore noto pubblicamente. Sebbene gli autori non siano a conoscenza di un attacco reale a tale caratteristica, ciò implica una forte correlazione tra le chiavi usate, e fornisce ad un attaccante dati interessanti da analizzare. Ulteriormente, l'impostazione è simile a quella utilizzata nello standard WEP, questo lascia intuire che c'è una grande esperienza nel campo dello \emph{statistical key recovery}.\\
Un altro problema di RC4 in OSGP è che è utilizzato con una autenticazione debole. Data la natura di uno stream cipher, un attaccante che entra in possesso di un \emph{ciphertext} può alterare bit in posizioni scelte. Grazie alla conoscenza dello spazio dei messaggi dello standard EN 14908, un attaccante può inviare messaggi alterati che sono validi con alta probabilità. Inoltre, egli può autenticare questo messaggi sfruttando la debolezza della funzione di digest descritta qui di seguito.
\subsubsection{Debole funzione di digest}
\subsubsection{Sicurezza di broadcast non definita}
\subsubsection{Utilizzo della chiave}
%-----------------------------------------------------------------------------
\section{Attacchi}
\subsection{Attacking Smart Meters and Smart Devices}
\subsection{False Data Injection}
\subsection{Time Delay Attack}
\subsection{Replay Attack}
\subsection{Jamming}
\subsection{Web Application}
\section{Rilevamento}
\subsection{Power Fingerprinting}
\subsection{Optimal Malicious Attack Construction and Robust Detection in Smart Grid Cyber Security Analysis}


\begin{thebibliography}{99}
\bibitem{standard14908} International Organization for Standardization. ISO/IEC 14908-1:2012:Information technology – Control network protocol – Part 1: Protocol stack, 2012.
\bibitem{wep1} Andreas Klein. \emph{Attacks on the RC4 stream cipher.} Des. Codes Cryptography, 48(3):269–286, 2008.
\bibitem{wep2} Erik Tews, Ralf-Philipp Weinmann, and Andrei Pyshkin. \emph{Breaking 104 bit WEP in less than 60 seconds.} In Sehun Kim, Moti Yung, and Hyung-Woo Lee, editors, Information Security Applications, 8th International Workshop, WISA 2007, Jeju Island, Korea, August 27-29, 2007, Revised Selected Papers, volume 4867 of Lecture Notes in Computer Science, pages 188–202. Springer, 2007.
\bibitem{aircrackng} Aircrack-ng. \textbf{http://www.aircrack-ng.org/}
\bibitem{kali} Kali Linux. \textbf{https://www.kali.org/}
\end{thebibliography}
Il termine \textbf{``sicurezza"} si riferisce alle tecniche, ai processi e ai provvedimenti adottati per proteggere dati, reti di comunicazione, tecnologie informatiche e sistemi di calcolo da accessi non autorizzati o da attacchi. \newline
L'approccio tradizionale prevede che la maggior parte delle risorse a disposizione per mettere in sicurezza il sistema si focalizzi sulle componenti più cruciali e che le protegga dalle minacce più grandi e più note; questo meccanismo fa si che le componenti secondarie siano indifese e, inoltre, non protette da attacchi meno pericolosi. Tale approccio, però, risulta inefficiente nell'ambito della Smart Grid. \newline Per adattarsi al nuovo sistema, le organizzazioni promuovono un metodo più proattivo ed adattivo: il NIST, per esempio, ha recentemente pubblicato delle linee guida che consigliano uno spostamento verso il continuo monitoraggio e verso valutazioni real-time [ref].\newline \newline
La sicurezza della Smart Grid, in relazione al suo sviluppo, è un tema fortemente discusso: tutti concordano nel sostenere che la Smart Grid dovrebbe avere un modello di sicurezza robusto; il problema è che ci si trova dinanzi a due sfide: come poter rispondere ai requisiti richiesti e come poter applicare le numerose alternative esistenti quando si cerca di rendere sicuro un ambiente complesso come la Smart Grid.
\newline \newline
Quando si sente parlare di ``nuova tecnologia", di ``interconnessione" e di ``condivisione dei dati", subito ci si focalizza sui benefici e sulle nuove funzionalità che tali concetti portano con loro. C'è da considerare, però, anche i nuovi rischi che queste nuove funzionalità introducono all'interno del sistema.\newline Per questo motivo, lo scopo della sicurezza è quello di garantire che le funzionalità del sistema operino correttamente e siano protette da abusi. È importante sottolineare, però, che non esistono applicazioni, reti o sistemi completamente sicuri e le Smart Grid non sono un'eccezione. Sebbene ogni componente della nuova rete elettrica porti con se numerosi miglioramenti operazionali  o funzionali, introduce anche nuove vulnerabilità e rischi addizionali che, se non propriamente gestiti, possono portare il sistema ad essere esposto ad attacchi di varia natura.
\section{Cenni storici}

\section{Definire la sicurezza}
La sicurezza tradizionale fa affidamento sulla cosiddetta \textbf{CIA triad} [ref.], che ne costituisce il cuore. La CIA triad comprende tre concetti: confidentiality, integrity ed availability. \newline Una concezione più moderna, e più adatta all'ambiente della Smart Grid, prevede l'utilizzo del \textbf{Parkerian hexad}, proposto da Parker nel 2002 [ref.]. Tale modello propone, in aggiunta ai tre classici concetti precedenti, altri tre principi: control (o possession), authenticity ed usability (o utility). \newline All'interno di questi sei pilastri, è possibile trovare tutti i problemi relativi alla Smart Grid.
\begin{figure}[h]\centering{
  \includegraphics[scale=0.5, natwidth=674,natheight=679]{imgs/hexad.png}
  \caption{Parkerian hexad}
}
\end{figure}
\subsection{Confidentiality}
Tale concetto porta con sé una serie di problemi e di preoccupazioni relative alla trasmissione e alla memorizzazione di dati ricavati dalle operazioni della Smart Grid. Questo tipo di dati, infatti, è spesso ritenuto \textit{confidenziale}, nel senso che se fosse noto, avrebbe tutto il potenziale per causare danni alla sicurezza delle operazioni di tutto il sistema. \newline La confidenzialità, inoltre, può essere intesa anche in un'altra accezione: se i dati fossero noti alla concorrenza, per esempio,  quest'ultima potrebbe trarre un notevole vantaggio in uno specifico settore o in tutto il mercato. \newline A tali fattori si aggiungono altre nuove problematiche legate alla \textit{privacy del consumatore} e, quindi, dei suoi dati, che vengono fuori da meccanismi di metering quali l'AMI. Gli utenti, infatti, si aspettano che i consumi relativi alle loro abitazioni private rimangano confidenziali; se così non fosse, la disponiblità di tali informazioni insieme alla capacità di fare data mining, avrebbe il potenziale per creare significative preoccupazioni sulla privacy. \newline I punti della Smart Grid che introducono rischi per la confidenzialità, sono costituiti da tutte le locazioni in cui sono memorizzati i dati e da tutti i meccanismi di trasmissione delle informazioni. Per quanto riguarda i dati memorizzati, questi potrebbero essere letti, copiati e distribuiti a soggetti diversi dai destinatari. Per quanto riguarda la trasmissione, invece, sia su reti private che su reti pubbliche come Internet, i dati potrebbero essere intercettati, copiati e distribuiti. \newline La soluzione a tali problemi risiede nelle funzioni di \textit{cifratura dei dati} e di \textit{controllo degli accessi}. Fornendo l'appropriato livello di cifratura delle informazioni, quest'ultime possono essere protette da chiunque non sia il diretto destinatario. \newline Il controllo degli accessi, prevede che i dati siano protetti da coloro che hanno l'autorizzazione per accedere al sistema ma che, allo stesso tempo, non hanno bisogno di tali dati per svolgere il loro lavoro.

\subsection{Integrity}
L'integrità si riferisce all'abilità del sistema di evitare che le informazioni possano essere modificate da persone o da sistemi non autorizzati. \newline Se si rendono possibili meccanismi di modifica volontari quali la manipolazione dei dati, o anche involontari quali la loro corruzione, i sistemi riceveranno informazioni non accurate; a lungo andare ciò potrebbe avere un impatto negativo su tutte le operazioni e, in casi estremi, portare ad instabilità o compromettere del tutto la Smart Grid. \newline
I punti della nuova rete elettrica che introducono rischi per l'integrità, sono tutti quei punti che consentono il passaggio dei dati da un sistema ad un altro. Pertanto, la sicurezza di tali meccanismi di transizione è importante, ma ancora più importante è come il sistema che riceve i dati possa assicurarsi della validità di quest'ultimi: se i dati subiscono manipolazioni mentre sono in viaggio tra i due sistemi, il ricevente potrebbe prendere decisioni basate su tali informazioni (che risultano essere errate); se, invece, i dati sono soggetti a corruzione durante la loro transizione, ci si potrebbe trovare difronte ad un comportamento inaspettato del ricevente. In entrambi i casi, è evidente che l'integrità dei dati sia cruciale per assicurare la stabilità delle operazioni.   \newline
Le risposte ai problemi di integrità, possono essere trovate nei meccanismi di \textit{auditing}, di \textit{authorization}, di \textit{nonrepudiation}, e di \textit{message-signing}, che saranno trattati in seguito (vedi paragrafo 3.3, building blocks).


\subsection{Availability}
Molto spesso si tende ad utilizzare i concetti di reliability ed availability in maniera intercambiabile; in realtà, tali concetti hanno due significati diversi. La reliability, infatti, risponde alle seguenti domande: ``quanto spesso fallisce il sistema?", ``quanto è elastico?"; l'availability, invece, indica la disponibilità del sistema e, quindi, la capacità di compiere il lavoro che gli è stato assegnato, \textit{nel momento in cui se ne ha bisogno}. \newline Una porzione del sistema potrebbe essere attiva, eseguendo e processando i comandi, il 100\% del tempo, e pertanto molto affidabile ma, se le performance non sono adeguate ai bisogni della rete e operazioni critiche vengono ritardate o mancano, non si può dire che il sistema sia disponibile. \newline
I punti della Smart Grid che introducono rischi per la disponibilità sono troppi per poterli elencare: qualsiasi sistema, rete, dispositivo che gestisce le comunicazioni, processo per la gestione dei messaggi, e qualsiasi servizio invocato a qualsiasi livello applicativo e il suo sistema operativo sottostante sono un rischio per la disponibilità quando si trovano a dover gestire l'inoltro di un comando da un'estremità del sistema ad un'altra. Risolvere tale rischio è quasi tanto complicato quanto identificare le componenti del sistema che hanno il potenziale per impattare sulla disponibilità. La maggior parte delle soluzioni si affida a tecniche di ridondanza (clustering, bilanciamento del carico); il costo di tali metodologie, però, cresce in maniera proporzionale ai punti di fallimento che si identificano nel sistema. 

\subsection{Control}

\subsection{Authenticity}

\subsection{Usability}


\section{Building blocks}

\section{Minacce e loro impatto}

\section{Sforzo dello stato (?)}

\section{Compagnie}

\section{Servizi di terze parti}

